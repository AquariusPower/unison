\newif\iftextversion \textversionfalse
\newif\iffull \fullfalse
\newif\ifdraft  \drafttrue

\IfFileExists{texdirectives.tex}
   {\input{texdirectives}}
   {\typeout{== ERROR == The dynamically generated file `texdirectives.tex' couldn't be found. You should use `make` to build the manual for Unison (`unison-manual.pdf'). Don't call latex directly.}\stop}
\input{texdirectives}
\input{unisonversion}

\newcommand{\finish}[1]{\ifdraft{\ifhevea\red\else \large\bf\fi [#1]\ifhevea\fi}\fi}
\newcommand{\finishlater}[1]{}
\newcommand{\fortrevor}[1]{}

\newcommand{\CLIENT}{\iftextversion CLIENT \else {\em client}\fi}
\newcommand{\SERVER}{\iftextversion SERVER \else {\em server}\fi}

\newcommand{\showtt}[1]{%
  \ifhevea
    \iftextversion
      "#1"%
    \else
      {\large\tt #1}%
    \fi
  \else
    {\tt #1}%
  \fi
}

\makeatletter
\def\@opentoc#1{\begingroup
  \makeatletter
  \if@filesw \expandafter\newwrite\csname tf@#1\endcsname
             \immediate\openout \csname tf@#1\endcsname \jobname.#1\relax
  \fi \global\@nobreakfalse \endgroup}
\newcommand{\TABLEOFCONTENTS}{%
  \ifhevea
    \iftextversion\else
      \section*{Contents}
      \begin{quote}
      \documentclass{article}
\usepackage{alltt}
\usepackage{fullpage}
\usepackage{moreverb}
% \usepackage{hyperref}
\usepackage{hevea}

\newif\iftextversion \textversionfalse
\newif\iffull \fullfalse
\newif\ifdraft  \drafttrue

\IfFileExists{texdirectives.tex}
   {\input{texdirectives}}
   {\typeout{== ERROR == The dynamically generated file `texdirectives.tex' couldn't be found. You should use `make` to build the manual for Unison (`unison-manual.pdf'). Don't call latex directly.}\stop}
\input{texdirectives}
\input{unisonversion}

\newcommand{\finish}[1]{\ifdraft{\ifhevea\red\else \large\bf\fi [#1]\ifhevea\fi}\fi}
\newcommand{\finishlater}[1]{}
\newcommand{\fortrevor}[1]{}

\newcommand{\CLIENT}{\iftextversion CLIENT \else {\em client}\fi}
\newcommand{\SERVER}{\iftextversion SERVER \else {\em server}\fi}

\newcommand{\showtt}[1]{%
  \ifhevea
    \iftextversion
      "#1"%
    \else
      {\large\tt #1}%
    \fi
  \else
    {\tt #1}%
  \fi
}

\makeatletter
\def\@opentoc#1{\begingroup
  \makeatletter
  \if@filesw \expandafter\newwrite\csname tf@#1\endcsname
             \immediate\openout \csname tf@#1\endcsname \jobname.#1\relax
  \fi \global\@nobreakfalse \endgroup}
\newcommand{\TABLEOFCONTENTS}{%
  \ifhevea
    \iftextversion\else
      \section*{Contents}
      \begin{quote}
      \documentclass{article}
\usepackage{alltt}
\usepackage{fullpage}
\usepackage{moreverb}
% \usepackage{hyperref}
\usepackage{hevea}

\input{local}
\fulltrue

%\newcommand{\NT}[1]{\(\langle\)\textit{#1}\(\rangle\)}
\newcommand{\NT}[1]{\textit{#1}}
\newcommand{\ARG}[1]{\texttt{\textit{#1}}}

%%%%%%%%%%%%%%%%%%%%%%%%%%%%%%%%%%%%%%%%%%%%%%%%%%%%%%%%%%%%%%%%%%%%%%
%%%%%%%%%%%%%%%%%%%%%%%%%%%%%%%%%%%%%%%%%%%%%%%%%%%%%%%%%%%%%%%%%%%%%%
\begin{document}

\ifhevea\begin{rawhtml}<div id="manualbody">\end{rawhtml}\fi

\ifhevea\else\bigskip\fi%
\ifdraft%
\begin{center}%
{\Huge \ifhevea\red\fi DraftDraftDraftDraft}%
\end{center}%
\ifhevea\else \bigskip \fi
\fi

\ifhevea\begin{rawhtml}<div id="manualheader">\end{rawhtml}%
\else \thispagestyle{empty}
\fi%
\SNIP{About Unison}{about}%
\iftextversion
  \section*{Unison File Synchronizer
%%   \\
%%   \ONEURL{http://www.cis.upenn.edu/\home{bcpierce}/unison}
  \\
  Version
  \unisonversion
  }
\else%
  \ifhevea\else \vspace*{2in} \fi%
  \begin{center}%
  \Huge{\ifhevea\black\else\bf \fi Unison File  Synchronizer}%
%%  \ifhevea \\ \else \\[2ex] \fi
%%   \large
%%   \ONEURL{http://www.cis.upenn.edu/\home{bcpierce}/unison}
  \ifhevea \\ \else \\[2ex] \fi%
  \huge {\ifhevea\black\else\bf \fi User Manual and Reference Guide}%
  \ifhevea \\ \else \\[6ex] \fi%
  \LARGE%
  Version \unisonversion \\[4ex] %
  % \today %
  \large Copyright 1998-2015, 2017, Benjamin C. Pierce
  \end{center}%
\fi%
%
%
\ifhevea\begin{rawhtml}</div>\end{rawhtml}\fi

\ifhevea\else\newpage\fi
\TABLEOFCONTENTS
\ifhevea\else\newpage\fi

\SECTION{Overview}{overview}{ }

\input{short}

\ifhevea\else\bigskip\fi

% \begin{quote}
% {\bf\ifhevea\red\fi Warning:} The current implementation of Unison is
% considered beta-test software.  It is in daily use by quite a few
% people, but there are still undoubtedly some bugs.  If you choose to
% use it to synchronize important data, please pay careful attention
% to what it is doing!  Also, the installation/setup procedure is not
% yet as smooth as we want it to be.
% \end{quote}


\SECTION{Preface}{intro}{ }

\TOPSUBSECTION{People}{people}

\URL{http://www.cis.upenn.edu/\home{bcpierce}/}{Benjamin Pierce} leads the
Unison project.
%
The current version of Unison was designed and implemented by
    \URL{http://www.research.att.com/\home{trevor}/}{Trevor Jim},
    \URL{http://www.cis.upenn.edu/\home{bcpierce}/}{Benjamin Pierce},
and
    \URL{http://www.pps.jussieu.fr/\home{vouillon}/}{J\'{e}r\^{o}me Vouillon},
with
    \URL{http://alan.petitepomme.net/}{Alan Schmitt},
    {Malo Denielou},
    \URL{http://www.brics.dk/\home{zheyang}/}{Zhe Yang},
    Sylvain Gommier, and
    Matthieu Goulay.
%
The Mac user interface was started by Trevor Jim and enormously improved by
Ben Willmore.
%
Our implementation of the
  \URL{http://samba.org/rsync/}{rsync}
  protocol was built by
  \URL{http://www.eecs.harvard.edu/\home{nr}/}{Norman Ramsey}
  and Sylvain Gommier.  It is based on
  \URL{http://samba.anu.edu.au/\home{tridge}/}{Andrew Tridgell}'s
  \URL{http://samba.anu.edu.au/\home{tridge}/phd\_thesis.pdf}{thesis work}
  and inspired by his
  \URL{http://samba.org/rsync/}{rsync}
  utility.
% \finish{Our low-level fingerprinting implementation uses an algorithm
% by Michael Rabin and incorporates some coding tricks from Andrei
% Broder and Mike Burrows.}
%
The mirroring and merging functionality was implemented by
  Sylvain Roy, improved by Malo Denielou, and improved yet further by
  St\'ephane Lescuyer.
%
 \URL{http://wwwfun.kurims.kyoto-u.ac.jp/\home{garrigue}/}{Jacques Garrigue}
 contributed the original Gtk version of the user
  interface; the Gtk2 version was built by Stephen Tse.
%
Sundar Balasubramaniam helped build a prototype implementation of
an earlier synchronizer in Java.
\URL{http://www.cis.upenn.edu/\home{ishin}/}{Insik Shin}
and
\URL{http://www.cis.upenn.edu/\home{lee}/}{Insup Lee} contributed design
ideas to this implementation.
\URL{http://research.microsoft.com/\home{fournet}/}{Cedric Fournet}
contributed to an even earlier prototype.

\TOPSUBSECTION{Mailing Lists and Bug Reporting}{lists}

\input{contactsbody}

\TOPSUBSECTION{Development Status}{status}

Unison is no longer under active development as a research
project.  (Our research efforts are now focused on a follow-on
project called Boomerang, described at
\ONEURL{http://www.cis.upenn.edu/\home{bcpierce}/harmony}.)
At this point, there is no one whose job it is to maintain Unison,
fix bugs, or answer questions.

However, the original developers are all still using Unison daily.  It
will continue to be maintained and supported for the foreseeable future,
and we will occasionally release new versions with bug fixes, small
improvements, and contributed patches.

Reports of bugs affecting correctness or safety are of interest to many
people and will generally get high priority.  Other bug reports will be
looked at as time permits.  Bugs should be reported to the users list at
\UNISONUSERS.

Feature requests are welcome, but will probably just be added to the
ever-growing todo list.  They should also be sent to \UNISONUSERS.

Patches are even more welcome.  They should be sent to
\UNISONHACKERS.
(Since safety and robustness are Unison's most important properties,
patches will be held to high standards of clear design and clean coding.)
If you want to contribute to Unison, start by downloading the developer
tarball from the download page.  For some details on how the code is
organized, etc., see the file {\tt CONTRIB}.

\TOPSUBSECTION{Copying}{copying}

This file is part of Unison.

    Unison is free software: you can redistribute it and/or modify
    it under the terms of the GNU General Public License as published by
    the Free Software Foundation, either version 3 of the License, or
    (at your option) any later version.

    Unison is distributed in the hope that it will be useful,
    but WITHOUT ANY WARRANTY; without even the implied warranty of
    MERCHANTABILITY or FITNESS FOR A PARTICULAR PURPOSE.  See the
    GNU General Public License for more details.

    The GNU Public License can be found at
    \ONEURL{http://www.gnu.org/licenses}.  A copy is also included in the
    Unison source distribution in the file {\tt COPYING}.

\TOPSUBSECTION{Acknowledgements}{ack}

Work on Unison has been supported by the National Science Foundation
under grants CCR-9701826 and ITR-0113226, {\em Principles and Practice of
  Synchronization}, and by University of Pennsylvania's Institute for
Research in Cognitive Science (IRCS).

\SECTION{Installation}{install}{install}

Unison is designed to be easy to install.  The following sequence of
steps should get you a fully working installation in a few minutes.  If
you run into trouble, you may find the suggestions on the
\SHOWURL{http://www.cis.upenn.edu/\home{bcpierce}/unison/faq.html}{Frequently Asked
Questions page} helpful.  Pre-built binaries are available for a
variety of platforms.

Unison can be used with either of two user interfaces:
\begin{enumerate}
\item a simple textual interface, suitable for dumb terminals (and
running from scripts), and
\item a more sophisticated graphical interface, based on Gtk2 (on
       Linux/Windows) or the native UI framework (on OSX).
\end{enumerate}

You will need to install a copy of Unison on every machine that you
want to synchronize.  However, you only need the version with a
graphical user interface (if you want a GUI at all) on the machine
where you're actually going to display the interface (the \CLIENT{}
machine).  Other machines that you synchronize with can get along just
fine with the textual version.


\SUBSECTION{Downloading Unison}{download}

The Unison download site lives under
\ONEURL{http://www.cis.upenn.edu/\home{bcpierce}/unison}.

If a pre-built binary of Unison is available for the client machine's
architecture, just download it and put it somewhere in your search
path (if you're going to invoke it from the command line) or on your
desktop (if you'll be click-starting it).

The executable file for the graphical version (with a name including
\verb|gtkui|) actually provides {\em both} interfaces: the graphical one
appears by default, while the textual interface can be selected by including
\verb|-ui text| on the command line.  The \verb|textui| executable
provides just the textual interface.

If you don't see a pre-built executable for your architecture, you'll
need to build it yourself.  See \sectionref{building}{Building Unison from Scratch}.
There are also a small number of contributed ports to other
architectures that are not maintained by us.  See the
\SHOWURL{http://www.cis.upenn.edu/\home{bcpierce}/unison/download.html}{Contributed
Ports page} to check what's available.

Check to make sure that what you have downloaded is really executable.
Either click-start it, or type \showtt{unison -version} at the command
line.

Unison can be used in three different modes: with different directories on a
single machine, with a remote machine over a direct socket connection, or
with a remote machine using {\tt ssh} for authentication and secure
transfer.  If you intend to use the last option, you may need to install
{\tt ssh}; see \sectionref{ssh}{Installing Ssh}.

\SUBSECTION{Running Unison}{afterinstall}

Once you've got Unison installed on at least one system, read
\sectionref{tutorial}{Tutorial} of the user manual (or type \showtt{unison -doc
  tutorial}) for instructions on how to get started.


\SUBSECTION{Upgrading}{upgrading}

Upgrading to a new version of Unison is as simple as throwing away the old
binary and installing the new one.

Before upgrading, it is a good idea to run the {\em old} version one last
time, to make sure all your replicas are completely synchronized.  A new
version of Unison will sometimes introduce a different format for the
archive files used to remember information about the previous state of the
replicas.  In this case, the old archive will be ignored (not deleted --- if
you roll back to the previous version of Unison, you will find the old
archives intact), which means that any differences between the replicas will
show up as conflicts that need to be resolved manually.


\SUBSECTION{Building Unison from Scratch}{building}

If a pre-built image is not available, you will need to compile it from
scratch; the sources are available from the same place as the binaries.

In principle, Unison should work on any platform to which OCaml has been
ported and on which the \verb|Unix| module is fully implemented.  It has
been tested on many flavors of Windows (98, NT, 2000, XP) and Unix (OS X,
Solaris, Linux, FreeBSD), and on both 32- and 64-bit architectures.


\SUBSUBSECTION{Unix}{build-unix}

Unison can be built with or without a graphical user interface (GUI). The
build system will decide automatically depending on the libraries installed
on your system, but you can also type {\tt make UISTYLE=text} to build
Unison without GUI.

You'll need the Objective Caml compiler,
available from \ONEURL{http://caml.inria.fr}.  OCaml is available from most
package managers
Building and installing OCaml
on Unix systems is very straightforward; just follow the instructions in the
distribution.  You'll probably want to build the native-code compiler in
addition to the bytecode compiler, as Unison runs much faster when compiled
to native code, but this is not absolutely necessary.
%
(Quick start: on many systems, the following sequence of commands will
get you a working and installed compiler: first do {\tt make world opt},
then {\tt su} to root and do {\tt make install}.)

You'll also need the GNU {\tt make} utility, which is standard on most Unix
systems.  Unison's build system is not parallelizable, so don't use flags
that cause it to start processes in parallel (e.g. -j).

Once you've got OCaml installed, grab a copy of the Unison sources, unzip
and untar them, change to the new \showtt{unison} directory, and type ``{\tt
  make UISTYLE=text}''.  The result should be an executable file called
\showtt{unison}.  Type \showtt{./unison} to make sure the program is
executable.  You should get back a usage message.

If you want to build the graphical user interface, you will need to install
some additional things:
\begin{itemize}
\item The Gtk2 development libraries (package {\tt libgtk2.0-dev} on debian
based systems).
\item OCaml bindings for Gtk2. Install them from your software repositories
(package {\tt liblablgtk2-ocaml} on debian based systems). Also available
from \ONEURL{http://wwwfun.kurims.kyoto-u.ac.jp/soft/olabl/lablgtk.html}.
\item Pango, a text rendering library and a part of Gtk2. On some systems
(e.g. Ubuntu) the bindings between Pango and OCaml need to be installed
explicitly (package {\tt liblablgtk-extras-ocaml-dev} on Ubuntu).
\end{itemize}
Type {\tt make src} to build Unison. If Gtk2 is available on the system,
Unison with a GUI will be built automatically.

Put the \verb|unison| executable somewhere in your search path, either by
adding the Unison directory to your PATH variable or by copying the
executable to some standard directory where executables are stored.  Or just
type {\tt make install} to install Unison to {\tt \$HOME/bin/unison}.

\SUBSUBSECTION{Mac OS X}{build-osx}

To build the text-only user interface, follow the instructions above for
building on Unix systems.  You should do this first, even if you are also
planning on building the GUI, just to make sure it works.

To build the basic GUI version, you'll first need to download and install
the XCode developer tools from Apple.  Once this is done, just type {\tt
  make} in the {\tt src} directory, and if things go well you
should get an application that you can move from {\tt
  uimac14/build/Default/Unison.app} to wherever you want it.

\SUBSUBSECTION{Windows}{build-win}

Although the binary distribution should work on any version of Windows,
some people may want to build Unison from scratch on those systems too.

\paragraph{Bytecode version:} The simpler but slower compilation option
to build a Unison executable is to build a bytecode version.  You need
first install Windows version of the OCaml compiler (version 3.07 or
later, available from \ONEURL{http://caml.inria.fr}).  Then grab a copy
of Unison sources and type
\begin{verbatim}
       make NATIVE=false
\end{verbatim}
to compile the bytecode.  The result should be an executable file called
\verb|unison.exe|.

\paragraph{Native version:} Building a more efficient, native version of
Unison on Windows requires a little more work.  See the file {\tt
  INSTALL.win32} in the source code distribution.


\SUBSUBSECTION{Installation Options}{build-opts}

The \verb|Makefile| in the distribution includes several switches that
can be used to control how Unison is built.  Here are the most useful
ones:
\begin{itemize}
\item Building with \verb|NATIVE=true| uses the native-code OCaml
compiler, yielding an executable that will run quite a bit faster. We use
this for building distribution versions.
\item Building with \verb|make DEBUGGING=true| generates debugging
symbols.
\item Building with \verb|make STATIC=true| generates a (mostly)
statically linked executable.  We use this for building distribution
versions, for portability.
\end{itemize}
%\finish{Any other important ones?}


\SECTION{Tutorial}{tutorial}{tutorial}

%\finish{Put a pointer somewhere in here to the typical profile in the
%  reference section.}

\SUBSECTION{Preliminaries}{prelim}

Unison can be used with either of two user interfaces:
\begin{enumerate}
\item a straightforward textual interface and
\item a more sophisticated graphical interface
\end{enumerate}
The textual interface is more convenient for running from scripts and
works on dumb terminals; the graphical interface is better for most
interactive use.  For this tutorial, you can use either.  If you are running
Unison from the command line, just typing {\tt unison}
will select either the text or the graphical interface, depending on which
has been selected as default when the executable you are running was
built.  You can force the text interface even if graphical is the default by
adding {\tt -ui text}.
The other command-line arguments to both versions are identical.

The graphical version can also be run directly by clicking on its icon, but
this may require a little set-up (see \sectionref{click}{Click-starting
  Unison}).  For this tutorial, we assume that you're starting it from the
command line.

Unison can synchronize files and directories on a single machine, or
between two machines on a network.  (The same program runs on both
machines; the only difference is which one is responsible for
displaying the user interface.)  If you're only interested in a
single-machine setup, then let's call that machine the \CLIENT{}.  If
you're synchronizing two machines, let's call them \CLIENT{} and
\SERVER.

\SUBSECTION{Local Usage}{local}

Let's get the client machine set up first and see how to synchronize
two directories on a single machine.

Follow the instructions in \sectionref{install}{Installation} to either
download or build an executable version of Unison, and install it
somewhere on your search path.  (If you just want to use the textual user
interface, download the appropriate textui binary.  If you just want to
the graphical interface---or if you will use both interfaces [the gtkui
binary actually has both compiled in]---then download the gtkui binary.)

Create a small test directory {\tt a.tmp} containing a couple of files
and/or subdirectories, e.g.,
\begin{verbatim}
       mkdir a.tmp
       touch a.tmp/a a.tmp/b
       mkdir a.tmp/d
       touch a.tmp/d/f
\end{verbatim}
Copy this directory to b.tmp:
\begin{verbatim}
       cp -r a.tmp b.tmp
\end{verbatim}

Now try synchronizing {\tt a.tmp} and {\tt b.tmp}.  (Since they are
identical, synchronizing them won't propagate any changes, but Unison
will remember the current state of both directories so that it will be
able to tell next time what has changed.)  Type:
\begin{verbatim}
       unison a.tmp b.tmp
\end{verbatim}
(You may need to add \verb|-ui text|, depending how your unison binary was built.)

\begin{textui}
You should see a message notifying you that all the files are actually
equal and then get returned to the command line.
\end{textui}

\begin{tkui}
You should get a big empty window with a message at the bottom
notifying you that all files are identical.  Choose the Exit item from
the File menu to get back to the command line.
\end{tkui}

Next, make some changes in a.tmp and/or b.tmp.  For example:
\begin{verbatim}
        rm a.tmp/a
        echo "Hello" > a.tmp/b
        echo "Hello" > b.tmp/b
        date > b.tmp/c
        echo "Hi there" > a.tmp/d/h
        echo "Hello there" > b.tmp/d/h
\end{verbatim}
Run Unison again:
\begin{verbatim}
       unison a.tmp b.tmp
\end{verbatim}

This time, the user interface will display only the files that have
changed.  If a file has been modified in just one
replica, then it will be displayed with an arrow indicating the
direction that the change needs to be propagated.  For example,
\begin{verbatim}
                 <---  new file   c  [f]
\end{verbatim}
\noindent
indicates that the file {\tt c} has been modified only in the second
replica, and that the default action is therefore to propagate the new
version to the first replica.  To {\bf f}ollow Unison's recommendation,
press the ``f'' at the prompt.

If both replicas are modified and their contents are different, then
the changes are in conflict: \texttt{<-?->} is displayed to indicate
that Unison needs guidance on which replica should override the
other.
\begin{verbatim}
     new file  <-?->  new file   d/h  []
\end{verbatim}
By default, neither version will be propagated and both
replicas will remain as they are.

If both replicas have been modified but their new contents are the same
(as with the file {\tt b}), then no propagation is necessary and
nothing is shown.  Unison simply notes that the file is up to date.

These display conventions are used by both versions of the user
interface.  The only difference lies in the way in which Unison's
default actions are either accepted or overridden by the user.

\begin{textui}
The status of each modified file is displayed, in turn.
When the copies of a file in the two replicas are not identical, the
user interface will ask for instructions as to how to propagate the
change.  If some default action is indicated (by an arrow), you can
simply press Return to go on to the next changed file.  If you want to
do something different with this file, press ``\verb|<|'' or ``\verb|>|'' to force
the change to be propagated from right to left or from left to right,
or else press ``\verb|/|'' to skip this file and leave both replicas alone.
When it reaches the end of the list of modified files, Unison will ask
you one more time whether it should proceed with the updates that have
been selected.

When Unison stops to wait for input from the user, pressing ``\verb|?|''
will always give a list of possible responses and their meanings.
\end{textui}

\begin{tkui}
The main window shows all the files that have been modified in either
{\tt a.tmp} or {\tt b.tmp}.  To override a default action (or to select
an action in the case when there is no default), first select the file, either
by clicking on its name or by using the up- and down-arrow keys.  Then
press either the left-arrow or ``\verb|<|'' key (to cause the version in b.tmp to
propagate to a.tmp) or the right-arrow or ``\verb|>|'' key (which makes the a.tmp
version override b.tmp).

Every keyboard command can also be invoked from the menus at the top
of the user interface.  (Conversely, each menu item is annotated with
its keyboard equivalent, if it has one.)

When you are satisfied with the directions for the propagation of changes
as shown in the main window, click the ``Go'' button to set them in
motion.  A check sign will be displayed next to each filename
when the file has been dealt with.
\end{tkui}


\SUBSECTION{Remote Usage}{remote}

Next, we'll get Unison set up to synchronize replicas on two different
machines.

Follow the instructions in the Installation section to download or
build an executable version of Unison on the server machine, and
install it somewhere on your search path.  (It doesn't matter whether
you install the textual or graphical version, since the copy of Unison on
the server doesn't need to display any user interface at all.)

It is important that the version of Unison installed on the server
machine is the same as the version of Unison on the client machine.
But some flexibility on the version of Unison at the client side can
be achieved by using the \verb|-addversionno| option; see
\sectionref{prefs}{Preferences}.

Now there is a decision to be made.  Unison provides two methods for
communicating between the client and the server:
\begin{itemize}
\item {\em Remote shell method}: To use this method, you must have
  some way of invoking remote commands on the server from the client's
  command line, using a facility such as \verb|ssh|.
  This method is more convenient (since there is no need to manually
  start a ``unison server'' process on the server) and also more
  secure (especially if you use \verb|ssh|).

\item {\em Socket method}: This method requires only that you can get
  TCP packets from the client to the server and back.  A draconian
  firewall can prevent this, but otherwise it should work anywhere.
\end{itemize}

Decide which of these you want to try, and continue with
\sectionref{rshmeth}{Remote Shell Method} or
\sectionref{socketmeth}{Socket Method}, as appropriate.


\SUBSECTION{Remote Shell Method}{rshmeth}

The standard remote shell facility on Unix systems is \verb|ssh|, which provides the
same functionality as the older \verb|rsh| but much better security.  Ssh is available from
\ONEURL{http://www.openssh.org}.  See section~\ref{ssh-win}
for installation instructions for the Windows version.

Running
\verb|ssh| requires some coordination between the client and server
machines to establish that the client is allowed to invoke commands on
the server; please refer to the \verb|ssh| documentation
for information on how to set this up.  The examples in this section
use \verb|ssh|, but you can substitute \verb|rsh| for \verb|ssh| if
you wish.

First, test that we can invoke Unison on the server from the client.
Typing
\begin{alltt}
        ssh \NT{remotehostname} unison -version
\end{alltt}
should print the same version information as running
\begin{verbatim}
        unison -version
\end{verbatim}
locally on the client.  If remote execution fails, then either
something is wrong with your ssh setup (e.g., ``permission denied'')
or else the search path that's being used when executing commands on
the server doesn't contain the \verb|unison| executable (e.g.,
``command not found'').

Create a test directory {\tt a.tmp} in your home directory on the client
machine.

Test that the local unison client can start and connect to the
remote server.  Type
\begin{alltt}
          unison -testServer a.tmp ssh://\NT{remotehostname}/a.tmp
\end{alltt}

Now cd to your home directory and type:
\begin{verbatim}
          unison a.tmp ssh://remotehostname/a.tmp
\end{verbatim}
The result should be that the entire directory {\tt a.tmp} is propagated
from the client to your home directory on the server.

After finishing the first synchronization, change a few files and try
synchronizing again.  You should see similar results as in the local
case.

If your user name on the server is not the same as on the client, you
need to specify it on the command line:
\begin{verbatim}
          unison a.tmp ssh://username@remotehostname/a.tmp
\end{verbatim}

\noindent {\it Notes:}
\begin{itemize}
\item If you want to put \verb|a.tmp| some place other than your home
directory on the remote host, you can give an absolute path for it by
adding an extra slash between \verb|remotehostname| and the beginning
of the path:
\begin{verbatim}
          unison a.tmp ssh://remotehostname//absolute/path/to/a.tmp
\end{verbatim}

\item You can give an explicit path for the \verb|unison| executable
  on the server by using the command-line option \showtt{-servercmd
    /full/path/name/of/unison} or adding
  \showtt{servercmd=/full/path/name/of/unison} to your profile (see
  \sectionref{profile}{Profiles}).  Similarly, you can specify a
  explicit path for the \verb|ssh| program using the \showtt{-sshcmd}
  option.
  Extra arguments can be passed to \verb|ssh| by setting the
  \verb|-sshargs| preference.
\end{itemize}


\SUBSECTION{Socket Method}{socketmeth}

\begin{quote}
  {\bf\ifhevea\red\fi Warning:} The socket method is
  insecure: not only are the texts of your changes transmitted over
  the network in unprotected form, it is also possible for anyone in
  the world to connect to the server process and read out the contents
  of your filesystem!  (Of course, to do this they must understand the
  protocol that Unison uses to communicate between client and server,
  but all they need for this is a copy of the Unison sources.)  The socket
  method is provided only for expert users with specific needs; everyone
  else should use the \verb|ssh| method.
\end{quote}

To run Unison over a socket connection, you must start a Unison
daemon process on the server.  This process runs continuously,
waiting for connections over a given socket from client machines
running Unison and processing their requests in turn.

To start the daemon, type
\begin{verbatim}
       unison -socket NNNN
\end{verbatim}
on the server machine, where {\tt NNNN} is the socket number that the
daemon should listen on for connections from clients.  ({\tt NNNN} can
be any large number that is not being used by some other program; if
\texttt{NNNN} is already in use, Unison will exit with an error
message.)  Note that paths specified by the client will be interpreted
relative to the directory in which you start the server process; this
behavior is different from the ssh case, where the path is relative to
your home directory on the server.

Create a test directory {\tt a.tmp} in your home directory on the
client machine.  Now type:
\begin{alltt}
       unison a.tmp socket://\NT{remotehostname}:NNNN/a.tmp
\end{alltt}
The result should be that the entire directory {\tt a.tmp} is
propagated from the client to the server (\texttt{a.tmp} will be
created on the server in the directory that the server was started
from).
%
After finishing the first synchronization, change a few files and try
synchronizing again.  You should see similar results as in the local
case.

Since the socket method is not used by many people, its functionality is
rather limited.  For example, the server can only deal with one client at a
time.


\SUBSECTION{Using Unison for All Your Files}{usingit}

Once you are comfortable with the basic operation of Unison, you may
find yourself wanting to use it regularly to synchronize your commonly
used files.  There are several possible ways of going about this:

\begin{enumerate}
\item Synchronize your whole home directory, using the Ignore facility
(see \sectionref{ignore}{Ignoring Paths})
to avoid synchronizing temporary files and things that only belong on
one host.
\item Create a subdirectory called {\tt shared} (or {\tt current}, or
whatever) in your home directory on each host, and put all the files
you want to synchronize into this directory.
\item Create a subdirectory called {\tt shared} (or {\tt current}, or
whatever) in your home directory on each host, and put {\em links to}
all the files you want to synchronize into this directory.  Use the
{\tt follow} preference (see \sectionref{symlinks}{Symbolic Links}) to make
Unison treat these links as transparent.
\item Make your home directory the root of the synchronization, but
tell Unison to synchronize only some of the files and subdirectories
within it on any given run.  This can be accomplished by using the {\tt -path} switch
on the command line:
\begin{alltt}
       unison /home/\NT{username} ssh://\NT{remotehost}//home/\NT{username} -path shared
\end{alltt}
The {\tt -path} option can be used as many times as needed, to
synchronize several files or subdirectories:
\begin{alltt}
       unison /home/\NT{username} ssh://\NT{remotehost}//home/\NT{username} \verb|\|
          -path shared \verb|\|
          -path pub \verb|\|
          -path .netscape/bookmarks.html
\end{alltt}
These \verb|-path| arguments can also be put in your preference file.
See \sectionref{prefs}{Preferences} for an example.
\end{enumerate}

Most people find that they only need to maintain a profile (or
profiles) on one of the hosts that they synchronize, since Unison is
always initiated from this host.  (For example, if you're
synchronizing a laptop with a fileserver, you'll probably always run
Unison on the laptop.)  This is a bit different from the usual
situation with asymmetric mirroring programs like \verb|rdist|, where
the mirroring operation typically needs to be initiated from the
machine with the most recent changes.  \sectionref{profile}{Profiles}
covers the syntax of Unison profiles, together with some sample profiles.

Some tips on improving Unison's performance can be found on the
\SHOWURL{http://www.cis.upenn.edu/\home{bcpierce}/unison/faq.html}{Frequently
  Asked Questions page}.

\SUBSECTION{Using Unison to Synchronize More Than Two Machines}{usingmultiple}

Unison is designed for synchronizing pairs of replicas.  However, it is
possible to use it to keep larger groups of machines in sync by performing
multiple pairwise synchronizations.

If you need to do this, the most reliable way to set things up is to
organize the machines into a ``star topology,'' with one machine designated
as the ``hub'' and the rest as ``spokes,'' and with each spoke machine
synchronizing only with the hub.  The big advantage of the star topology is
that it eliminates the possibility of confusing ``spurious conflicts''
arising from the fact that a separate archive is maintained by Unison for
every pair of hosts that it synchronizes.


\SUBSECTION{Going Further}{further}

On-line documentation for the various features of Unison
can be obtained either by typing
\begin{verbatim}
        unison -doc topics
\end{verbatim}
\noindent
at the command line, or by selecting the Help menu in the graphical
user interface.
\iftextversion
The same information is also available in a typeset User's
Manual (HTML or PostScript format) through
\ONEURL{http://www.cis.upenn.edu/\home{bcpierce}/unison}.
\else
The on-line information and the printed manual are essentially identical.
\fi

If you use Unison regularly, you should subscribe to one of the mailing
lists, to receive announcements of new versions.  See
\sectionref{lists}{Mailing Lists and Bug Reporting}.

\SECTION{Basic Concepts}{basics}{basics}

To understand how Unison works, it is necessary to discuss a few
straightforward concepts.
%
These concepts are developed more rigorously and at more length in a number
of papers, available at \ONEURL{http://www.cis.upenn.edu/\home{bcpierce}/papers}.
But the informal presentation here should be enough for most users.


\SUBSECTION{Roots}{roots}

A replica's {\em root} tells Unison where to find a set of files to be
synchronized, either on the local machine or on a remote host.
For example,
\begin{alltt}
      \NT{relative/path/of/root}
\end{alltt}
\noindent
specifies a local root relative to the directory where Unison is
started, while
\begin{alltt}
      /\NT{absolute/path/of/root}
\end{alltt}
\noindent
specifies a root relative to the top of the local filesystem,
independent of where Unison is running.  Remote roots can begin with
\verb|ssh://|,
\verb|rsh://|
to indicate that the remote server should be started with rsh or ssh:
\begin{alltt}
      ssh://\NT{remotehost}//\NT{absolute/path/of/root}
      rsh://\NT{user}@\NT{remotehost}/\NT{relative/path/of/root}
\end{alltt}
If the remote server is already running (in the socket mode), then the syntax
\begin{alltt}
      socket://\NT{remotehost}:\NT{portnum}//\NT{absolute/path/of/root}
      socket://\NT{remotehost}:\NT{portnum}/\NT{relative/path/of/root}
\end{alltt}
\noindent
is used to specify the hostname and the port that the client Unison should
use to contact it.

The syntax for roots is based on that of URIs (described in RFC 2396).
The full grammar is:
\begin{alltt}
  \NT{replica} ::= [\NT{protocol}:]//[\NT{user}@][\NT{host}][:\NT{port}][/\NT{path}]
           |  \NT{path}

  \NT{protocol} ::= file
            |  socket
            |  ssh
            |  rsh

  \NT{user} ::= [-_a-zA-Z0-9]+

  \NT{host} ::= [-_a-zA-Z0-9.]+

  \NT{port} ::= [0-9]+
\end{alltt}
When \verb|path| is given without any protocol prefix, the protocol is
assumed to be \verb|file:|.  Under Windows, it is possible to
synchronize with a remote directory using the \verb|file:| protocol over
the Windows Network Neighborhood.  For example,
\begin{verbatim}
       unison foo //host/drive/bar
\end{verbatim}
\noindent
synchronizes the local directory \verb|foo| with the directory
\verb|drive:\bar| on the machine \verb|host|, provided that \verb|host|
is accessible via Network Neighborhood.  When the \verb|file:| protocol
is used in this way, there is no need for a Unison server to be running
on the remote host.  However, running Unison this way is only a good
idea if the remote host is reached by a very fast network connection,
since the full contents of every file in the remote replica will have to
be transferred to the local machine to detect updates.

The names of roots are {\em canonized} by Unison before it uses them
to compute the names of the corresponding archive files, so {\tt
  //saul//home/bcpierce/common} and {\tt //saul.cis.upenn.edu/common}
will be recognized as the same replica under different names.

\SUBSECTION{Paths}{paths}

A {\em path} refers to a point {\em within} a set of files being
synchronized; it is specified relative to the root of the replica.

Formally, a path is just a sequence of names, separated by \verb|/|.
Note that the path separator character is always a forward slash, no
matter what operating system Unison is running on.  Forward slashes
are converted to backslashes as necessary when paths are converted to
filenames in the local filesystem on a particular host.
%
(For example, suppose that we run Unison on a Windows system, synchronizing
the local root \verb|c:\pierce| with the root
\verb|ssh://saul.cis.upenn.edu/home/bcpierce| on a Unix server.  Then
the path \verb|current/todo.txt| refers to the file
\verb|c:\pierce\current\todo.txt| on the client and
\verb|/home/bcpierce/current/todo.txt| on the server.)

The empty path (i.e., the empty sequence of names) denotes the whole
replica.  Unison displays the empty path as ``\verb|[root]|.''

If \verb|p| is a path and \verb|q| is a path beginning with \verb|p|, then
\verb|q| is said to be a {\em descendant} of \verb|p|.  (Each path is also a
descendant of itself.)


\SUBSECTION{What is an Update?}{updates}

The {\em contents} of a path \verb|p| in a particular replica could be a
file, a directory, a symbolic link, or absent (if \verb|p| does not
refer to anything at all in that replica).  More specifically:
\begin{itemize}
\item If \verb|p| refers to an ordinary file, then the
contents of \verb|p| are the actual contents of this file (a string of bytes)
plus the current permission bits of the file.
\item If \verb|p| refers to a symbolic link, then the contents of \verb|p|
are just the string specifying where the link points.
\item If \verb|p| refers to a directory, then the
contents of \verb|p| are just the token ``DIRECTORY'' plus the current
permission bits of the directory.
\item If \verb|p| does not refer to anything in this replica, then the
contents of \verb|p| are the token ``ABSENT.''
\end{itemize}
Unison keeps a record of the contents of each path after each
successful synchronization of that path (i.e., it remembers the
contents at the last moment when they were the same in the two
replicas).

We say that a path is {\em updated} (in some replica) if its current
contents are different from its contents the last time it was successfully
synchronized.  Note that whether a path is updated has nothing to do with
its last modification time---Unison considers only the contents when
determining whether an update has occurred.  This means that touching a file
without changing its contents will {\em not} be recognized as an update.  A
file can even be changed several times and then changed back to its original
contents; as long as Unison is only run at the end of this process, no
update will be recognized.

What Unison actually calculates is a close approximation to this
definition; see \sectionref{caveats}{Caveats and Shortcomings}.

\SUBSECTION{What is a Conflict?}{conflicts}

A path is said to be {\em conflicting} if the following conditions all hold:
\begin{enumerate}
\item it has been updated in one replica,
\item it or any of its descendants has been updated in the other
  replica,
and
\item its contents in the two replicas are not identical.
\end{enumerate}

\finishlater{Note that this isn't precisely what we implement, in the
  case of directory permission changes!}


\SUBSECTION{Reconciliation}{recon}

Unison operates in several distinct stages:
\begin{enumerate}
\item On each host, it compares its archive file (which records
the state of each path in the replica when it was last synchronized)
with the current contents of the replica, to determine which paths
have been updated.
\item It checks for ``false conflicts'' --- paths that have been
updated on both replicas, but whose current values are identical.
These paths are silently marked as synchronized in the archive files
in both replicas.
\item It displays all the updated paths to the user.  For updates that
do not conflict, it suggests a default action (propagating the new
contents from the updated replica to the other).  Conflicting updates
are just displayed.  The user is given an opportunity to examine the
current state of affairs, change the default actions for
nonconflicting updates, and choose actions for conflicting updates.
\item It performs the selected actions, one at a time.  Each action is
performed by first transferring the new contents to a temporary file
on the receiving host, then atomically moving them into place.
\item It updates its archive files to reflect the new state of the
replicas.
\end{enumerate}

\TOPSUBSECTION{Invariants}{failures}

Given the importance and delicacy of the job that it performs, it is
important to understand both what a synchronizer does under normal
conditions and what can happen under unusual conditions such as system
crashes and communication failures.

% Unison deals with two sorts of information: the two replicas
% themselves and its own memory of the ``last synchronized state'' of
% each path in the replicas.  The latter is what allows it to detect
% correctly which replica is new when a file been updated.  Roughly,
% the sequence of actions that occur when Unison runs is:
% \begin{enumerate}
% \item It reads a private archive file stored with each replica
% and checks which paths on each replica have been updated.
% Technically, a path has been updated if its contents in a replica are
% different from the contents of that replica at the end of the last
% synchronization in which that path was successfully synchronized ---
% i.e., the last time the two replicas were equal at that path at the
% end of a run of Unison.  The ``contents'' of a path can be either a
% file, a directory, or nothing at all, so deleting a file or changing a
% directory to a file count as updates to the contents at that path.

% For efficiency, Unison does not try to calculate the set of updated
% paths exactly: it will sometimes falsely detect a change in a path
% whose contents have actually not changed (this can happen, for
% example, when the file's modification time has been changed, for some
% reason).  As long as this path has not been modified in the other
% replica, this ``conservativity'' in update detection is invisible to
% the user.  If the other replica {\em has} been modified, however, a
% ``false conflict'' may be reported.

% \item It combines the lists of paths that (may) have been updated in
% the two replicas, assigns default actions to those where the change
% was in one replica only, and records a conflict for those that were
% changed in both replicas.

% \item The current contents of the paths on this list are then
% compared, to see if they actually differ.  (This is done by comparing
% fingerprints, not transferring the whole files.)  Paths whose contents
% are actually identical are marked as synchronized and deleted from the
% list.

% \item The remaining paths are displayed to the user, who then has an
% opportunity to change the default actions and choose actions for
% conflicting paths.

% \item When this process is finished, the selected changes are actually
% propagated between the replicas.

% \item Finally, Unison updates its internal state, marking as
% synchronized all the files for which changes were successfully
% propagated.
% \end{enumerate}

Unison is careful to protect both its internal state and the state of
the replicas at every point in this process.  Specifically, the
following guarantees are enforced:
\begin{itemize}
\item At every moment, each path in each replica has either (1) its {\em
  original} contents (i.e., no change at all has been made to this
path), or (2) its {\em correct} final contents (i.e., the value that the
user expected to be propagated from the other replica).
\item At every moment, the information stored on disk about Unison's
private state can be either (1) unchanged, or (2) updated to reflect
those paths that have been successfully synchronized.
\end{itemize}
The upshot is that it is safe to interrupt Unison at any time, either
manually or accidentally.  [Caveat: the above is {\em almost} true there
are occasionally brief periods where it is not (and, because of
shortcoming of the Posix filesystem API, cannot be); in particular, when
it is copying a file onto a directory or vice versa, it must first move
the original contents out of the way.  If Unison gets
interrupted during one of these periods, some manual cleanup may be
required.  In this case, a file called {\tt DANGER.README} will be left
in your home directory, containing information about the operation that
was interrupted. The next time you try to run Unison, it will notice this
file and warn you about it.]

If an interruption happens while it is propagating updates, then there
may be some paths for which an update has been propagated but which
have not been marked as synchronized in Unison's archives.  This is no
problem: the next time Unison runs, it will detect changes to these
paths in both replicas, notice that the contents are now equal, and
mark the paths as successfully updated when it writes back its private
state at the end of this run.

If Unison is interrupted, it may sometimes leave temporary working files
(with suffix \verb|.tmp|) in the replicas.  It is safe to delete these
files.  Also, if the \verb|backups| flag is set, Unison will
leave around old versions of files that it overwrites, with names like
\verb|file.0.unison.bak|.  These can be deleted safely when they are no
longer wanted.

Unison is not bothered by clock skew between the different hosts on
which it is running.  It only performs comparisons between timestamps
obtained from the same host, and the only assumption it makes about
them is that the clock on each system always runs forward.

If Unison finds that its archive files have been deleted (or that the
archive format has changed and they cannot be read, or that they don't
exist because this is the first run of Unison on these particular
roots), it takes a conservative approach: it behaves as though the
replicas had both been completely empty at the point of the last
synchronization.  The effect of this is that, on the first run, files
that exist in only one replica will be propagated to the other, while
files that exist in both replicas but are unequal will be marked as
conflicting.

Touching a file without changing its contents should never affect whether or
not Unison does an update. (When running with the fastcheck preference set
to true---the default on Unix systems---Unison uses file modtimes for a
quick first pass to tell which files have definitely not changed; then, for
each file that might have changed, it computes a fingerprint of the file's
contents and compares it against the last-synchronized contents. Also, the
\verb|-times| option allows you to synchronize file times, but it does not
cause identical files to be changed; Unison will only modify the file
times.)

It is safe to ``brainwash'' Unison by deleting its archive files
{\em on both replicas}.  The next time it runs, it will assume that
all the files it sees in the replicas are new.

It is safe to modify files while Unison is working.  If Unison
discovers that it has propagated an out-of-date change, or that the
file it is updating has changed on the target replica, it will signal
a failure for that file.  Run Unison again to propagate the latest
change.
\finishlater{There are some race conditions. We should probably talk about them.}

Changes to the ignore patterns from the user interface (e.g., using
the `i' key) are immediately reflected in the current profile.


\SUBSECTION{Caveats and Shortcomings}{caveats}

Here are some things to be careful of when using Unison.

\begin{itemize}
\item In the interests of speed, the update detection algorithm may
  (depending on which OS architecture that you run Unison on)
  actually use an approximation to the definition given in
  \sectionref{updates}{What is an Update?}.

  In particular, the Unix
  implementation does not compare the actual contents of files to their
  previous contents, but simply looks at each file's inode number and
  modtime; if neither of these have changed, then it concludes that the
  file has not been changed.

  Under normal circumstances, this approximation is safe, in the sense
  that it may sometimes detect ``false updates'' but will never miss a real
  one.  However, it is possible to fool it, for example by using
  \verb|retouch| to change a file's modtime back to a time in the past.
  \finishlater{One user---Marcus Mottl---claimed that it could also
  happen if we use
  memory mapped I/O, but this is not clear}

\item If you synchronize between a single-user filesystem and a shared
Unix server, you should pay attention to your permission bits: by
default, Unison will synchronize permissions verbatim, which may leave
group-writable files on the server that could be written over by a lot of
people.

You can control this by setting your \verb|umask| on both computers to
something like 022, masking out the ``world write'' and ``group write''
permission bits.

Unison does not synchronize the \verb|setuid| and \verb|setgid| bits, for
security.

\item The graphical user interface is single-threaded.  This
means that if Unison is performing some long-running operation, the
display will not be repainted until it finishes.  We recommend not
trying to do anything with the user interface while Unison is in the
middle of detecting changes or propagating files.

\item Unison does not understand hard links.

\item It is important to be a little careful when renaming directories
containing {\tt ignore}d files.

For example, suppose Unison is synchronizing directory A between the two
machines called the ``local'' and the ``remote'' machine; suppose directory
A contains a subdirectory D; and suppose D on the local machine contains a
file or subdirectory P that matches an ignore directive in the profile used
to synchronize. Thus path A/D/P exists on the local machine but not on the
remote machine.

 If D is renamed to D' on the remote machine, and this change is
 propagated to the local machine, all such files or subdirectories P
 will be deleted.  This is because Unison sees the rename as a delete and a
 separate create: it deletes the old directory (including the ignored files)
 and creates a new one ({\em not} including the ignored files, since they
 are completely invisible to it).
\end{itemize}



\SECTION{Reference Guide}{reference}{ }

This section covers the features of Unison in detail.

\TOPSUBSECTION{Running Unison}{running}

There are several ways to start Unison.
\begin{itemize}
\item Typing ``{\tt unison \NT{profile}}'' on the command line.  Unison
will look for a file \texttt{\NT{profile}.prf} in the \verb|.unison|
directory.  If this file does not specify a pair of roots, Unison will
prompt for them and add them to the information specified by the profile.
\item Typing ``{\tt unison \NT{profile} \NT{root1} \NT{root2}}'' on the command
line.
In this case, Unison will use {\tt \NT{profile}}, which should not contain
any {\tt root} directives.
\item Typing ``{\tt unison \NT{root1} \NT{root2}}'' on the command line.  This
has the same effect as typing ``{\tt unison default \NT{root1} \NT{root2}}.''
\item Typing just ``{\tt unison}'' (or invoking Unison by clicking on
a desktop icon).  In this case, Unison will ask for the profile to use
for synchronization (or create a new one, if necessary).
\end{itemize}

% \finish{Need to check that the text UI actually works this way.  (It
%   doesn't prompt, for sure, but it should.)}

\SUBSECTION{The {\tt .unison} Directory}{unisondir}

Unison stores a variety of information in a private directory on each
host.  If the environment variable {\tt UNISON} is defined, then its
value will be used as the name of this directory.  If {\tt UNISON} is
not defined, then the name of the directory depends on which
operating system you are using.  In Unix, the default is to use
{\tt \$HOME/.unison}.
In Windows, if the environment variable
{\tt USERPROFILE} is defined, then the directory will be
{\tt \$USERPROFILE$\backslash$.unison};
otherwise if {\tt HOME} is defined, it will be
{\tt \$HOME$\backslash$.unison};
otherwise, it will be
{\tt c:$\backslash$.unison}.
On OS X,
{\tt \$HOME/.unison} will be used if it is present, but
{\tt \$HOME/Library/Application Support/Unison} will be created and used by
default.

The archive file for each replica is found in the {\tt .unison}
directory on that replica's host.  Profiles (described below) are
always taken from the {\tt .unison} directory on the client host.

Note that Unison maintains a completely different set of archive files
for each pair of roots.

We do not recommend synchronizing the whole {\tt .unison} directory, as this
will involve frequent propagation of large archive files.  It should be safe
to do it, though, if you really want to.  Synchronizing just the profile
files in the {\tt .unison} directory is definitely OK.


\SUBSECTION{Archive Files}{archives}

The name of the archive file on each replica is calculated from
\begin{itemize}
\item the {\em canonical names} of all the hosts (short names like
  \verb|saul| are converted into full addresses like \verb|saul.cis.upenn.edu|),
\item the paths to the replicas on all the hosts (again, relative
  pathnames, symbolic links, etc.\ are converted into full, absolute paths), and
\item an internal version number that is changed whenever a new Unison
  release changes the format of the information stored in the archive.
\end{itemize}
This method should work well for most users.  However, it is occasionally
useful to change the way archive names are generated.  Unison provides
two ways of doing this.

The function that finds the canonical hostname of the local host (which
is used, for example, in calculating the name of the archive file used to
remember which files have been synchronized) normally uses the
\verb|gethostname| operating system call.  However, if the environment
variable \verb|UNISONLOCALHOSTNAME| is set, its value will be used
instead.  This makes it easier to use Unison in situations where a
machine's name changes frequently (e.g., because it is a laptop and gets
moved around a lot).

A more powerful way of changing archive names is provided by the
\verb|rootalias| preference.  The preference file may contain any number of
lines of the form:
\begin{alltt}
    rootalias = //\NT{hostnameA}//\NT{path-to-replicaA} -> //\NT{hostnameB}/\NT{path-to-replicaB}
\end{alltt}
When calculating the name of the archive files for a given pair of roots,
Unison replaces any root that matches the left-hand side of any rootalias
rule by the corresponding right-hand side.

So, if you need to relocate a root on one of the hosts, you can add a
rule of the form:
\begin{alltt}
    rootalias = //\NT{new-hostname}//\NT{new-path} -> //\NT{old-hostname}/\NT{old-path}
\end{alltt}
Note that root aliases are case-sensitive, even on case-insensitive file
systems.

{\em Warning}: The \verb|rootalias| option is dangerous and should only
be used if you are sure you know what you're doing.  In particular, it
should only be used if you are positive that either (1) both the original
root and the new alias refer to the same set of files, or (2) the files
have been relocated so that the original name is now invalid and will
never be used again.  (If the original root and the alias refer to
different sets of files, Unison's update detector could get confused.)
%
After introducing a new \verb|rootalias|, it is a good idea to run Unison
a few times interactively (with the \verb|batch| flag off, etc.) and
carefully check that things look reasonable---in particular, that update
detection is working as expected.


\SUBSECTION{Preferences}{prefs}

Many details of Unison's behavior are configurable by user-settable
``preferences.''

Some preferences are boolean-valued; these are often called {\em flags}.
Others take numeric or string arguments, indicated in the preferences
list by {\tt n} or {\tt xxx}.  Some string arguments take the backslash as
an escape to include the next character literally; this is mostly useful
to escape a space or the backslash; a trailing backslash is ignored and is
useful to protect a trailing whitespace in the string that would otherwise
be trimmed.  Most of the string preferences can be given several times;
the arguments are accumulated into a list internally.

There are two ways to set the values of preferences: temporarily, by
providing command-line arguments to a particular run of Unison, or
permanently, by adding commands to a {\em profile} in the {\tt .unison}
directory on the client host.  The order of preferences (either on the
command line or in preference files) is not significant.  On the command
line, preferences and other arguments (the profile name and roots) can be
intermixed in any order.

To set the value of a preference {\tt p} from the command line, add an
argument {\tt -p} (for a boolean flag) or {\tt -p n} or {\tt -p xxx} (for
a numeric or string preference) anywhere on the command line.  To set a
boolean flag to \verb|false| on the command line, use {\tt -p=false}.

Here are all the preferences supported by Unison.  This list can be
  obtained by typing {\tt unison -help}.
\begin{quote}
\verbatiminput{prefs.tmp}
\end{quote}
Here, in more detail, is what they do.  Many are discussed in greater detail
in other sections of the manual.
%
\input{prefsdocs.tmp}


\SUBSECTION{Profiles}{profile}

A {\em profile} is a text file that specifies permanent settings for
roots, paths, ignore patterns, and other preferences, so that they do
not need to be typed at the command line every time Unison is run.
Profiles should reside in the \verb|.unison| directory on the client
machine.  If Unison is started with just one argument \ARG{name} on
the command line, it looks for a profile called \texttt{\ARG{name}.prf} in
the \verb|.unison| directory.  If it is started with no arguments, it
scans the \verb|.unison| directory for files whose names end in
\verb|.prf| and offers a menu (provided that the Unison executable is compiled with the graphical user interface).  If a file named \verb|default.prf| is
found, its settings will be offered as the default choices.

To set the value of a preference {\tt p} permanently, add to the
appropriate profile a line of the form
\begin{verbatim}
        p = true
\end{verbatim}
for a boolean flag or
\begin{verbatim}
        p = <value>
\end{verbatim}
for a preference of any other type.

Whitespaces around {\tt p} and {\tt xxx} are ignored.
A profile may also include blank lines and lines beginning
with {\tt \#}; both are ignored.

When Unison starts, it first reads the profile and then the command
line, so command-line options will override settings from the
profile.

Profiles may also include lines of the form \texttt{include
  \ARG{name}}, which will cause the file \ARG{name} (or
\texttt{\ARG{name}.prf}, if \ARG{name} does not exist in the
\verb+.unison+ directory) to be read at the point, and included as if
its contents, instead of the \texttt{include} line, was part of the
profile.  Include lines allows settings common to several profiles to
be stored in one place.  In \ARG{name} the backslash is an escape
character.

A profile may include a preference `\texttt{label = \ARG{desc}}' to
provide a description of the options selected in this profile.  The
string \ARG{desc} is listed along with the profile name in the profile
selection dialog, and displayed in the top-right corner of the main
Unison window in the graphical user interface.

The graphical user-interface also supports one-key shortcuts for commonly
used profiles.  If a profile contains a preference of the form
%
`\texttt{key = \ARG{n}}', where \ARG{n} is a single digit, then
pressing this digit key will cause Unison to immediately switch to
this profile and begin synchronization again from scratch.  In this
case, all actions that have been selected for a set of changes
currently being displayed will be discarded.


\SUBSECTION{Sample Profiles}{profileegs}

\SUBSUBSECTION{A Minimal Profile}{minimalprofile}

Here is a very minimal profile file, such as might be found in {\tt
  .unison/default.prf}:
\begin{verbatim}
    # Roots of the synchronization
    root = /home/bcpierce
    root = ssh://saul//home/bcpierce

    # Paths to synchronize
    path = current
    path = common
    path = .netscape/bookmarks.html
\end{verbatim}

\SUBSUBSECTION{A Basic Profile}{basicprofile}

Here is a more sophisticated profile, illustrating some other useful
features.
\begin{verbatim}
    # Roots of the synchronization
    root = /home/bcpierce
    root = ssh://saul//home/bcpierce

    # Paths to synchronize
    path = current
    path = common
    path = .netscape/bookmarks.html

    # Some regexps specifying names and paths to ignore
    ignore = Name temp.*
    ignore = Name *~
    ignore = Name .*~
    ignore = Path */pilot/backup/Archive_*
    ignore = Name *.o
    ignore = Name *.tmp

    # Window height
    height = 37

    # Keep a backup copy of every file in a central location
    backuplocation = central
    backupdir = /home/bcpierce/backups
    backup = Name *
    backupprefix = $VERSION.
    backupsuffix =

    # Use this command for displaying diffs
    diff = diff -y -W 79 --suppress-common-lines

    # Log actions to the terminal
    log = true
\end{verbatim}

\SUBSUBSECTION{A Power-User Profile}{powerprofile}

When Unison is used with large replicas, it is often convenient to be
able to synchronize just a part of the replicas on a given run (this
saves the time of detecting updates in the other parts).  This can be
accomplished by splitting up the profile into several parts --- a common
part containing most of the preference settings, plus one ``top-level''
file for each set of paths that need to be synchronized.  (The {\tt
  include} mechanism can also be used to allow the same set of preference
settings to be used with different roots.)

The collection
of profiles implementing this scheme might look as follows.
%
The file {\tt default.prf} is empty except for an {\tt include}
directive:
\begin{verbatim}
    # Include the contents of the file common
    include common
\end{verbatim}
Note that the name of the common file is {\tt common}, not {\tt
  common.prf}; this prevents Unison from offering {\tt common} as one of
the list of profiles in the opening dialog (in the graphical UI).

The file {\tt common} contains the real preferences:
\begin{verbatim}
    # Roots of the synchronization
    root = /home/bcpierce
    root = ssh://saul//home/bcpierce

    # (... other preferences ...)

    # If any new preferences are added by Unison (e.g. 'ignore'
    # preferences added via the graphical UI), then store them in the
    # file 'common' rather than in the top-level preference file
    addprefsto = common

    # Names and paths to ignore:
    ignore = Name temp.*
    ignore = Name *~
    ignore = Name .*~
    ignore = Path */pilot/backup/Archive_*
    ignore = Name *.o
    ignore = Name *.tmp
\end{verbatim}
Note that there are no {\tt path} preferences in {\tt common}.  This
means that, when we invoke Unison with the default profile (e.g., by
typing '{\tt unison default}' or just '{\tt unison}' on the command
line), the whole replicas will be synchronized.  (If we {\em never} want
to synchronize the whole replicas, then {\tt default.prf} would instead
include settings for all the paths that are usually synchronized.)

To synchronize just part of the replicas, Unison is invoked with an
alternate preference file---e.g., doing '{\tt unison workingset}', where the
preference file {\tt workingset.prf} contains
\begin{verbatim}
    path = current/papers
    path = Mail/inbox
    path = Mail/drafts
    include common
\end{verbatim}
causes Unison to synchronize just the listed subdirectories.

The {\tt key} preference can be used in combination with the graphical UI
to quickly switch between different sets of paths.  For example, if the
file {\tt mail.prf} contains
\begin{verbatim}
    path = Mail
    batch = true
    key = 2
    include common
\end{verbatim}
then pressing 2 will cause Unison to look for updates in the {\tt Mail}
subdirectory and (because the {\tt batch} flag is set) immediately
propagate any that it finds.


\SUBSECTION{Keeping Backups}{backups}

When Unison overwrites (or deletes) a file or directory while propagating changes from
the other replica, it can keep the old version around as a backup.  There
are several preferences that control precisely where these backups are
stored and how they are named.

To enable backups, you must give one or more \verb|backup| preferences.
Each of these has the form
\begin{verbatim}
    backup = <pathspec>
\end{verbatim}
where \verb|<pathspec>| has the same form as for the \verb|ignore|
preference.  For example,
\begin{verbatim}
    backup = Name *
\end{verbatim}
causes Unison to keep backups of {\em all} files and directories.  The
\verb|backupnot| preference can be used to give a few exceptions: it
specifies which files and directories should {\em not} be backed up, even if
they match the \verb|backup| pathspec.

It is important to note that the \verb|pathspec| is matched against the path
that is being updated by Unison, not its descendants.  For example, if you
set \verb|backup = Name *.txt| and then delete a whole directory named
\verb|foo| containing some text files, these files will not be backed up
because Unison will just check that \verb|foo| does not match \verb|*.txt|.
Similarly, if the directory itself happened to be called \verb|foo.txt|,
then the whole directory and all the files in it will be backed up,
regardless of their names.

Backup files can be stored either {\em centrally} or {\em locally}.  This
behavior is controlled by the preference \verb|backuplocation|, whose value
must be either \verb|central| or \verb|local|.  (The default is
\verb|central|.)

When backups are stored locally, they are kept in the same
directory as the original.

When backups are stored centrally, the directory used to hold them is
controlled by the preference \verb|backupdir| and the
environment variable \verb|UNISONBACKUPDIR|.  (The environment variable is
checked first.)  If neither of these are set, then the directory
\verb|.unison/backup| in the user's home directory is used.

The preference \verb|maxbackups| controls how many previous versions of
each file are kept (including the current version).

By default, backup files are named \verb|.bak.VERSION.FILENAME|,
where \verb|FILENAME| is the original filename and \verb|VERSION| is the
backup number (1 for the most recent, 2 for the next most recent,
etc.).  This can be changed by setting the preferences \verb|backupprefix|
and/or \verb|backupsuffix|.  If desired, \verb|backupprefix| may include a
directory prefix; this can be used with \verb|backuplocation = local| to put all
backup files for each directory into a single subdirectory.  For example, setting
\begin{verbatim}
    backuplocation = local
    backupprefix = .unison/$VERSION.
    backupsuffix =
\end{verbatim}
will put all backups in a local subdirectory named \verb|.unison|.  Also,
note that the string \verb|$VERSION| in either \verb|backupprefix| or
\verb|backupsuffix| (it must appear in one or the other) is replaced by
the version number.  This can be used, for example, to ensure that backup
files retain the same extension as the originals.

For backward compatibility, the \verb|backups| preference is also supported.
%
It simply means \verb|backup = Name *| and \verb|backuplocation = local|.


\SUBSECTION{Merging Conflicting Versions}{merge}

Unison can invoke external programs to merge conflicting versions of a file.
The preference \verb|merge| controls this process.

The \verb|merge| preference may be given once or several times in a
preference file (it can also be given on the command line, of course, but
this tends to be awkward because of the spaces and special characters
involved).  Each instance of the preference looks like this:
\begin{verbatim}
    merge = <PATHSPEC> -> <MERGECMD>
\end{verbatim}
The \verb|<PATHSPEC>| here has exactly the same format as for the
\verb|ignore| preference (see \sectionref{pathspec}{Path Specification}).  For example,
using ``\verb|Name *.txt|'' as the \verb|<PATHSPEC>| tells Unison that this
command should be used whenever a file with extension \verb|.txt| needs to
be merged.

Many external merging programs require as inputs not just the two files that
need to be merged, but also a file containing the {\em last synchronized
  version}.  You can ask Unison to keep a copy of the last synchronized
version for some files using the \verb|backupcurrent| preference. This
preference is used in exactly the same way as \verb|backup| and its meaning
is similar, except that it causes backups to be kept of the {\em current}
contents of each file after it has been synchronized by Unison, rather than
the {\em previous} contents that Unison overwrote.  These backups are kept
on {\em both} replicas in the same place as ordinary backup files---i.e.
according to the \verb|backuplocation| and \verb|backupdir| preferences.
They are named like the original files if \verb|backupslocation| is set to
'central' and otherwise, Unison uses the \verb|backupprefix| and
\verb|backupsuffix| preferences and assumes a version number 000 for these
backups.

The \verb|<MERGECMD>| part of the preference specifies what external command
should be invoked to merge files at paths matching the \verb|<PATHSPEC>|.
Within this string, several special substrings are recognized; these will be
substituted with appropriate values before invoking a sub-shell to execute
the command.
\begin{itemize}
\item \relax\verb|CURRENT1| is replaced by the name of (a temporary copy of)
  the local variant of the file.
\item \relax\verb|CURRENT2| is replaced by the name of a temporary
  file, into which the contents of the remote variant of the file have
  been transferred by Unison prior to performing the merge.
\item \relax\verb|CURRENTARCH| is replaced by the name of the backed up copy
  of the original version of the file (i.e., the file saved by Unison
  if the current filename matches the path specifications for the
  \verb|backupcurrent| preference, as explained above), if one exists.
  If no archive exists and \relax\verb|CURRENTARCH| appears in the
  merge command, then an error is signalled.
\item \relax\verb|CURRENTARCHOPT| is replaced by the name of the backed up copy
  of the original version of the file (i.e., its state at the end of
  the last successful run of Unison), if one exists, or the empty
  string if no archive exists.
\item \relax\verb|NEW| is replaced by the name of a temporary file
  that Unison expects to be written by the merge program when it
  finishes, giving the desired new contents of the file.
\item \relax\verb|PATH| is replaced by the path (relative to the roots of
  the replicas) of the file being merged.
\item \relax\verb|NEW1| and \relax\verb|NEW2| are replaced by the names of temporary files
  that Unison expects to be written by the merge program when it
  is only able to partially merge the originals; in this case, \verb|NEW1|
  will be written back to the local replica and \verb|NEW2| to the remote
  replica; \verb|NEWARCH|, if present, will be used as the ``last common
  state'' of the replicas.  (These three options are provided for
  later compatibility with the Harmony data synchronizer.)
\item \relax\verb|BATCHMODE| is replaced according to the batch mode of
  Unison; if it is in \texttt{batch} mode, then a non empty string
  (``\verb|batch|'') is substituted, otherwise the empty string is substituted.
\end{itemize}
To accommodate the wide variety of programs that users might want to use for
merging, Unison checks for several possible situations when the merge
program exits:
\begin{itemize}
\item If the merge program exits with a non-zero status, then merge is
  considered to have failed and the replicas are not changed.
\item If the file \verb|NEW| has been created, it is written back to both
  replicas (and stored in the backup directory).  Similarly, if just the
  file \verb|NEW1| has been created, it is written back to both
  replicas.
\item If neither \verb|NEW| nor \verb|NEW1| have been created, then Unison
  examines the temporary files \verb|CURRENT1|  and \verb|CURRENT2| that
  were given as inputs to the merge program.  If either has been changed (or
  both have been changed in identical ways), then its new contents are written
  back to both replicas.  If either \verb|CURRENT1| or \verb|CURRENT2| has
  been {\em deleted}, then the contents of the other are written back to
  both replicas.
\item If the files \verb|NEW1|, \verb|NEW2|, and \verb|NEWARCH| have all
  been created, they are written back to the local replica, remote replica,
  and backup directory, respectively. If the files \verb|NEW1|, \verb|NEW2| have
  been created, but \verb|NEWARCH| has not, then these files are written back to the
  local replica and remote replica, respectively.  Also, if \verb|NEW1| and
  \verb|NEW2| have identical contents, then the same contents are stored as
  a backup (if the \verb|backupcurrent| preference is set for this path) to
  reflect the fact that the path is currently in sync.
  \item If \verb|NEW1| and \verb|NEW2| (resp. \verb|CURRENT1| and
  \verb|CURRENT2|) are created (resp. overwritten) with different contents
  but the merge command did not fail (i.e., it exited with status code 0),
  then we copy \verb|NEW1| (resp. \verb|CURRENT1|) to the other replica and
  to the archive.

  This behavior is a design choice made to handle the case where a merge
  command only synchronizes some specific contents between two files,
  skipping some irrelevant information (order between entries, for
  instance).  We assume that, if the merge command exits normally, then the
  two resulting files are ``as good as equal.'' (The reason we copy one on
  top of the other is to avoid Unison detecting that the files are unequal
  the next time it is run and trying again to merge them when, in fact, the
  merge program has already made them as similar as it is able to.)
\end{itemize}

You can disable a merge by setting a \verb|<MERGECMD>| that does nothing.  For
example you can override the merging of text files specified in a profile by
typing on the command line:
\begin{verbatim}
    unison profile -merge 'Name *.txt -> echo SKIP'
\end{verbatim}

If the \verb|confirmmerge| preference is set and Unison is not run in
batch mode, then Unison will always ask for confirmation before
actually committing the results of the merge to the replicas.

You can detect batch mode by testing \verb|BATCHMODE|; for
example to avoid a merge completely do nothing:
\begin{verbatim}
    merge = Name *.txt -> [ -z "BATCHMODE" ] && mergecmd CURRENT1 CURRENT2
\end{verbatim}

A large number of external merging programs are available.
For example, on Unix systems setting the \verb|merge| preference to
\begin{verbatim}
    merge = Name *.txt -> diff3 -m CURRENT1 CURRENTARCH CURRENT2
                            > NEW || echo "differences detected"
\end{verbatim}
\noindent
will tell Unison to use the external \verb|diff3| program for merging.
%
Alternatively, users of \verb|emacs| may find the following settings convenient:
\begin{verbatim}
    merge = Name *.txt -> emacs -q --eval '(ediff-merge-files-with-ancestor
                             "CURRENT1" "CURRENT2" "CURRENTARCH" nil "NEW")'
\end{verbatim}
\noindent
(These commands are displayed here on two lines to avoid running off the
edge of the page.  In your preference file, each command should be written on a
single line.)

Users running emacs under windows may find something like this useful:
\begin{verbatim}
   merge = Name * -> C:\Progra~1\Emacs\emacs\bin\emacs.exe -q --eval
                            "(ediff-files """CURRENT1""" """CURRENT2""")"
\end{verbatim}

Users running Mac OS X (you may need the Developer Tools installed to get
the {\tt opendiff} utility) may prefer
\begin{verbatim}
    merge = Name *.txt -> opendiff CURRENT1 CURRENT2 -ancestor CURRENTARCH -merge NEW
\end{verbatim}
Here is a slightly more involved hack.  The {\tt opendiff} program can
operate either with or without an archive file.  A merge command of this
form
\begin{verbatim}
    merge = Name *.txt ->
              if [ CURRENTARCHOPTx = x ];
              then opendiff CURRENT1 CURRENT2 -merge NEW;
              else opendiff CURRENT1 CURRENT2 -ancestor CURRENTARCHOPT -merge NEW;
              fi
\end{verbatim}
(still all on one line in the preference file!) will test whether an archive
file exists and use the appropriate variant of the arguments to {\tt
  opendiff}.

Linux users may enjoy this variant:
\begin{verbatim}
    merge = Name * -> kdiff3 -o NEW CURRENTARCHOPT CURRENT1 CURRENT2
\end{verbatim}

Ordinarily, external merge programs are only invoked when Unison is {\em
  not} running in batch mode.  To specify an external merge program that
should be used no matter the setting of the {\tt batch} flag, use the {\tt
  mergebatch} preference instead of {\tt merge}.

\begin{quote}
\it
Please post suggestions for other useful values of the
\verb|merge| preference to the {\tt unison-users} mailing list---we'd like
to give several examples here.
\end{quote}

\finishlater{
\SUBSECTION{Communicating with a Remote Server}{server}

If you can mount both filesystems on the same host, then you can
run with no server (note, though, that this won't be fast enough over
a phone line)..........
}

\SUBSECTION{The User Interface}{ui}

Both the textual and the graphical user interfaces are intended to be
mostly self-explanatory.  Here are just a few tricks:
\begin{itemize}
\item By default, when running on Unix the textual user interface will
try to put the terminal into the ``raw mode'' so that it reads the input a
character at a time rather than a line at a time.  (This means you can
type just the single keystroke ``\verb|>|'' to tell Unison to
propagate a file from left to right, rather than ``\verb|>| Enter.'')

There are some situations, though, where this will not work --- for
example, when Unison is running in a shell window inside Emacs.
Setting the \verb|dumbtty| preference will force Unison to leave the
terminal alone and process input a line at a time.
\end{itemize}

\SUBSECTION{Exit Code}{exit}

When running in the textual mode, Unison returns an exit status, which
describes whether, and at which level, the synchronization was successful.
The exit status could be useful when Unison is invoked from a script.
Currently, there are four possible values for the exit status:
\begin{itemize}
\item [0]: successful synchronization; everything is up-to-date now.
\item [1]: some files were skipped, but all file transfers were successful.
\item [2]: non-fatal failures occurred during file transfer.
\item [3]: a fatal error occurred, or the execution was interrupted.
\end{itemize}
The graphical interface does not return any useful information through the
exit status.

\SUBSECTION{Path Specification}{pathspec}
Several Unison preferences (e.g., \verb|ignore|/\verb|ignorenot|,
\verb|follow|, \verb|sortfirst|/\verb|sortlast|, \verb|backup|,
\verb|merge|, etc.)
specify individual paths or sets of paths.  These preferences share a
common syntax based on regular-expressions.  Each preference
is associated with a list of path patterns; the paths specified are those
that match any one of the path pattern.

\begin{itemize}
\item Pattern preferences can be given on the command line,
  or, more often, stored in profiles, using the same syntax as other preferences.
  For example, a profile line of the form
\begin{alltt}
             ignore = \ARG{pattern}
\end{alltt}
adds \ARG{pattern} to the list of patterns to be ignored.

\item Each \ARG{pattern} can have one of three forms.  The most
general form is a Posix extended regular expression introduced by the
keyword \verb|Regex|.  (The collating sequences and character classes of
full Posix regexps are not currently supported).
\begin{alltt}
                 Regex \ARG{regexp}
\end{alltt}
For convenience, three other styles of pattern are also recognized:
\begin{alltt}
                 Name \ARG{name}
\end{alltt}
matches any path in which the last component matches \ARG{name},
\begin{alltt}
                 Path \ARG{path}
\end{alltt}
matches exactly the path \ARG{path}, and
\begin{alltt}
                 BelowPath \ARG{path}
\end{alltt}
matches the path \ARG{path} and any path below.
%
The \ARG{name} and \ARG{path} arguments of the latter forms of
patterns are {\em not} regular expressions.  Instead,
standard ``globbing'' conventions can be used in \ARG{name} and
\ARG{path}:
\begin{itemize}
\item a \verb|*| matches any sequence of characters not including \verb|/|
(and not beginning with \verb|.|, when used at the beginning of a
\ARG{name})
\item a \verb|?| matches any single character except \verb|/| (and leading
  \verb|.|)
\item \verb|[xyz]| matches any character from the set $\{{\tt x},
  {\tt y}, {\tt z} \}$
\item \verb|{a,bb,ccc}| matches any one of \verb|a|, \verb|bb|, or
  \verb|ccc|.  (Be careful not to put extra spaces after the commas:
  these will be interpreted literally as part of the strings to be matched!)
\end{itemize}
\item
The path separator in path patterns is always the
forward-slash character ``/'' --- even when the client or server is
running under Windows, where the normal separator character is a
backslash.  This makes it possible to use the same set of path
patterns for both Unix and Windows file systems.
\end{itemize}
Some examples of path patterns appear in \sectionref{ignore}{Ignoring
  Paths}.

\SUBSECTION{Ignoring Paths}{ignore}

Most users of Unison will find that their replicas contain lots of
files that they don't ever want to synchronize --- temporary files,
very large files, old stuff, architecture-specific binaries, etc.
They can instruct Unison to ignore these paths using patterns
introduced in \sectionref{pathspec}{Path Specification}.

For example, the following pattern will make Unison ignore any
path containing the name \verb|CVS| or a name ending in \verb|.cmo|:
\begin{verbatim}
             ignore = Name {CVS,*.cmo}
\end{verbatim}
The next pattern makes Unison ignore the path \verb|a/b|:
\begin{verbatim}
             ignore = Path a/b
\end{verbatim}
Path patterns do {\em not} skip filenames beginning with \verb|.| (as Name
patterns do).  For example,
\begin{verbatim}
             ignore = Path */tmp
\end{verbatim}
will include \verb|.foo/tmp| in the set of ignore directories, as it is a
path, not a name, that is ignored.

The following pattern makes Unison ignore any path beginning with \verb|a/b|
and ending with a name ending by \verb|.ml|.
\begin{verbatim}
             ignore = Regex a/b/.*\.ml
\end{verbatim}
Note that regular expression patterns are ``anchored'': they must
match the whole path, not just a substring of the path.

Here are a few extra points regarding the \texttt{ignore} preference.
\begin{itemize}
\item If a directory is ignored, all its descendants will be too.

\item The user interface provides some convenient commands for adding
  new patterns to be ignored.  To ignore a particular file, select it
  and press ``{\tt i}''.  To ignore all files with the same extension,
  select it and press ``{\tt E}'' (with the shift key).  To ignore all
  files with the same name, no matter what directory they appear in,
  select it and press ``{\tt N}''.
%
These new patterns become permanent: they
are immediately added to the current profile on disk.

\item If you use the \verb|include| directive to include a common
collection of preferences in several top-level preference files, you will
probably also want to set the \verb|addprefsto| preference to the name of
this file.  This will cause any new ignore patterns that you add from
inside Unison to be appended to this file, instead of whichever top-level
preference file you started Unison with.

\item Ignore patterns can also be specified on the command line, if
you like (this is probably not very useful), using an option like
\verb|-ignore 'Name temp.txt'|.

\item Be careful about renaming directories containing ignored files.
Because Unison understands the rename as a delete plus a create, any ignored
files in the directory will be lost (since they are invisible to Unison and
therefore they do not get recreated in the new version of the directory).

\item There is also an \verb|ignorenot| preference, which specifies a set of
  patterns for paths that should {\em not} be ignored, even if they match an
  \verb|ignore| pattern.  However, the interaction of these two sets of
  patterns can be a little tricky.  Here is exactly how it works:
  \begin{itemize}
  \item Unison starts detecting updates from the root of the
  replicas---i.e., from the empty path.  If the empty path matches an
  \verb|ignore| pattern and does not match an \verb|ignorenot| pattern, then
  the whole replica will be ignored.  (For this reason, it is not a good
  idea to include \verb|Name *| as an \verb|ignore| pattern.  If you want to
  ignore everything except a certain set of files, use \verb|Name ?*|.)
  \item If the root is a directory, Unison continues looking for updates in
  all the immediate children of the root.  Again, if the name of some child matches an
  \verb|ignore| pattern and does not match an \verb|ignorenot| pattern, then
  this whole path {\em including everything below it} will be ignored.
  \item If any of the non-ignored children are directories, then the process
  continues recursively.
  \end{itemize}
\end{itemize}

\SUBSECTION{Symbolic Links}{symlinks}

Ordinarily, Unison treats symbolic links in Unix replicas as
``opaque'': it considers the contents of the link to be just the
string specifying where the link points, and it will propagate changes in
this string to the other replica.

It is sometimes useful to treat a symbolic link ``transparently,''
acting as though whatever it points to were physically {\em in} the
replica at the point where the symbolic link appears.  To tell Unison
to treat a link in this manner, add a line of the form
\begin{alltt}
             follow = \ARG{pathspec}
\end{alltt}
to the profile, where \ARG{pathspec} is a path pattern as described in
\sectionref{pathspec}{Path Specification}.

Windows file systems do not support symbolic links; Unison will refuse
to propagate an opaque symbolic link from Unix to Windows and flag the
path as erroneous.  When a Unix replica is to be synchronized with a
Windows system, all symbolic links should match either an
\verb|ignore| pattern or a \verb|follow| pattern.


\SUBSECTION{Permissions}{perms}

Synchronizing the permission bits of files is slightly tricky when two
different filesystems are involved (e.g., when synchronizing a Windows
client and a Unix server).  In detail, here's how it works:
\begin{itemize}
\item When the permission bits of an existing file or directory are
changed, the values of those bits that make sense on {\em both}
operating systems will be propagated to the other replica.  The other
bits will not be changed.
\item When a newly created file is propagated to a remote replica, the
permission bits that make sense in both operating systems are also
propagated.  The values of the other bits are set to default values
(they are taken from the current umask, if the receiving host is a
Unix system).
\item For security reasons, the Unix \verb|setuid| and \verb|setgid|
bits are not propagated.
\item The Unix owner and group ids are not propagated.  (What would
this mean, in general?)  All files are created with the owner and
group of the server process.
\end{itemize}


\SUBSECTION{Cross-Platform Synchronization}{crossplatform}

If you use Unison to synchronize files between Windows and Unix
systems, there are a few special issues to be aware of.

\textbf{Case conflicts.}  In Unix, filenames are case sensitive:
\texttt{foo} and \texttt{FOO} can refer to different files.  In
Windows, on the other hand, filenames are not case sensitive:
\texttt{foo} and \texttt{FOO} can only refer to the same file.  This
means that a Unix \texttt{foo} and \texttt{FOO} cannot be synchronized
onto a Windows system --- Windows won't allow two different files to
have the ``same'' name.  Unison detects this situation for you, and
reports that it cannot synchronize the files.

You can deal with a case conflict in a couple of ways.  If you need to
have both files on the Windows system, your only choice is to rename
one of the Unix files to avoid the case conflict, and re-synchronize.
If you don't need the files on the Windows system, you can simply
disregard Unison's warning message, and go ahead with the
synchronization; Unison won't touch those files.  If you don't want to
see the warning on each synchronization, you can tell Unison to ignore
the files (see \sectionref{ignore}{Ignoring Paths}).

\textbf{Illegal filenames.}  Unix allows some filenames that are
illegal in Windows.  For example, colons (`:') are not allowed in
Windows filenames, but they are legal in Unix filenames.  This means
that a Unix file \texttt{foo:bar} can't be synchronized to a Windows
system.  As with case conflicts, Unison detects this situation for
you, and you have the same options: you can either rename the Unix
file and re-synchronize, or you can ignore it.


\SUBSECTION{Slow Links}{speed}

Unison is built to run well even over relatively slow links such as
modems and DSL connections.

Unison uses the ``rsync protocol'' designed by Andrew Tridgell and Paul
Mackerras to greatly speed up transfers of large files in which only
small changes have been made.  More information about the rsync protocol
can be found at the rsync web site (\ONEURL{http://samba.anu.edu.au/rsync/}).

If you are using Unison with {\tt ssh}, you may get some speed
improvement by enabling {\tt ssh}'s compression feature.  Do this by
adding the option ``{\tt -sshargs -C}'' to the command line or ``{\tt
  sshargs = -C}'' to your profile.


\SUBSECTION{Making Unison Faster on Large Files}{speeding}

Unison's built-in implementation of the rsync algorithm makes transferring
updates to existing files pretty fast.  However, for whole-file copies of
newly created files, the built-in transfer method is not highly optimized.
Also, if Unison is interrupted in the middle of transferring a large file,
it will attempt to retransfer the whole thing on the next run.

These shortcomings can be addressed with a little extra work by telling
Unison to use an external file copying utility for whole-file transfers.
The recommended one is the standalone {\tt rsync} tool, which is available
by default on most Unix systems and can easily be installed on Windows
systems using Cygwin.

If you have {\tt rsync} installed on both hosts, you can make Unison use it
simply by setting the {\tt copythreshold} flag to something non-negative.
If you set it to 0, Unison will use the external copy utility for {\em all}
whole-file transfers.  (This is probably slower than letting Unison copy
small files by itself, but can be useful for testing.)  If you set it to a
larger value, Unison will use the external utility for all files larger than
this size (which is given in kilobytes, so setting it to 1000 will cause the
external tool to be used for all transfers larger than a megabyte).

If you want to use a different external copy utility, set both the {\tt
  copyprog} and {\tt copyprogrest} preferences---the former is used for
the first transfer of a file, while the latter is used when Unison sees a
partially transferred temp file on the receiving host.  Be careful here:
Your external tool needs to be instructed to copy files in place (otherwise
if the transfer is interrupted Unison will not notice that some of the data
has already been transferred, the next time it tries).  The default values
are:
\begin{verbatim}
   copyprog      =   rsync --inplace --compress
   copyprogrest  =   rsync --partial --inplace --compress
\end{verbatim}
You may also need to set the {\tt copyquoterem} preference.  When it is set
to {\tt true}, this causes Unison to add an extra layer of quotes to
the remote path passed to the external copy program. This is is needed by
rsync, for example, which internally uses an ssh connection, requiring an
extra level of quoting for paths containing spaces. When this flag is set to
{\tt default}, extra quotes are added if the value of {\tt copyprog}
contains the string {\tt rsync}.  The default value is {\tt default},
naturally.

If a {\em directory} transfer is interrupted, the next run of Unison will
automatically skip any files that were completely transferred before the
interruption.  (This behavior is always on: it does not depend on the
setting of the {\tt copythreshold} preference.)  Note, though, that the new
directory will not appear in the destination filesystem until everything has
been transferred---partially transferred directories are kept in a temporary
location (with names like {\tt .unison.DIRNAME....}) until the transfer is
complete.


\SUBSECTION{Fast Update Detection}{fastcheck}

If your replicas are large and at least one of them is on a Windows
system, you may find that Unison's default method for detecting changes
(which involves scanning the full contents of every file on every
sync---the only completely safe way to do it under Windows) is too slow.
Unison provides a preference {\tt fastcheck} that, when set to
\verb|true|, causes it to use file creation times as 'pseudo inode
numbers' when scanning replicas for updates, instead of reading the full
contents of every file.

When \verb|fastcheck| is set to \verb|no|,
Unison will perform slow checking---re-scanning the contents of each file
on each synchronization---on all replicas.  When \verb|fastcheck| is set
to \verb|default| (which, naturally, is the default), Unison will use
fast checks on Unix replicas and slow checks on Windows replicas.

This strategy may cause Unison to miss propagating an update if the
 modification time and length of the file are both unchanged
by the update.
However, Unison will never {\em overwrite} such an update with a change
from the other replica, since it always does a safe check for updates
just before propagating a change.  Thus, it is reasonable to use this
switch most of the time and occasionally run Unison once with {\tt
  fastcheck} set to \verb|no|, if you are worried that Unison may have
overlooked an update.

Fastcheck is (always) automatically disabled for files with extension
\verb|.xls| or \verb|.mpp|, to prevent Unison from being confused by the
habits of certain programs (Excel, in particular) of updating files without
changing their modification times.

\SUBSECTION{Mount Points and Removable Media}{mountpoints}

Using Unison removable media such as USB drives can be dangerous unless you
are careful.  If you synchronize a directory that is stored on removable
media when the media is not present, it will look to Unison as though the
whole directory has been deleted, and it will proceed to delete the
directory from the other replica---probably not what you want!

To prevent accidents, Unison provides a preference called
\verb|mountpoint|.  Including a line like
\begin{verbatim}
             mountpoint = foo
\end{verbatim}
in your preference file will cause Unison to check, after it finishes
detecting updates, that something actually exists at the path
\verb|foo| on both replicas; if it does not, the Unison run will
abort.

\SUBSECTION{Click-starting Unison}{click}

On Windows NT/2k/XP systems, the graphical version of Unison can be
invoked directly by clicking on its icon.  On Windows 95/98 systems,
click-starting also works, {\em as long as you are not using ssh}.
Due to an incompatibility with OCaml and Windows 95/98 that is not
under our control, you must start Unison from a DOS window in Windows
95/98 if you want to use ssh.

When you click on the Unison icon, two windows will be created:
Unison's regular window, plus a console window, which is used only for
giving your password to ssh (if you do not use ssh to connect, you can
ignore this window).  When your password is requested, you'll need to
activate the console window (e.g., by clicking in it) before typing.
If you start Unison from a DOS window, Unison's regular window will
appear and you will type your password in the DOS window you were
using.

To use Unison in this mode, you must first create a profile (see
\sectionref{profile}{Profiles}).  Use your favorite editor for this.


\appendix
\SECTION{Installing Ssh}{ssh}{ssh}

{\em Warning: These instructions may be out of date.  More current
  information can be found the
  \SHOWURL{http://alliance.seas.upenn.edu/~bcpierce/wiki/index.php?n=Main.UnisonFAQOSSpecific}{Unison
    Wiki}.}

Your local host will need just an ssh client; the remote host needs an
ssh server (or daemon), which is available on Unix systems.  Unison is
known to work with ssh version 1.2.27 (Unix) and version 1.2.14
(Windows); other versions may or may not work.

\SUBSECTION{Unix}{ssh-unix}

Most modern Unix installations come with \verb|ssh| pre-installed.

\SUBSECTION{Windows}{ssh-win}
Many Windows implementations of ssh only provide graphical interfaces,
but Unison requires an ssh client that it can invoke with a
command-line interface.  A suitable version of ssh can be installed as
follows.

\begin{enumerate}
\item Download an \verb|ssh| executable.

Warning: there are many implementations and ports of ssh for
Windows, and not all of them will work with Unison.  We have gotten
Unison to work with Cygwin's port of OpenSSH, and we suggest you try
that one first.  Here's how to install it:
\begin{enumerate}
\item First, create a new folder on your desktop to hold temporary
  installation files.  It can have any name you like, but in these
  instructions we'll assume that you call it \verb|Foo|.
\item Direct your web browser to www.cygwin.com, and click on the
  ``Install now!'' link.  This will download a file, \verb|setup.exe|;
  save it in the directory \verb|Foo|.  The file \verb|setup.exe| is a
  small program that will download the actual install files from
  the Internet when you run it.
\item Start \verb|setup.exe| (by double-clicking).  This brings up a
  series of dialogs that you will have to go through.  Select
  ``Install from Internet.''  For ``Local Package Directory'' select
  the directory \verb|Foo|.  For ``Select install root directory'' we
  recommend that you use the default, \verb|C:\cygwin|.  The next
  dialog asks you to select the way that you want to connect to the
  network to download the installation files; we have used ``Use IE5
  Settings'' successfully, but you may need to make a different
  selection depending on your networking setup.  The next dialog gives
  a list of mirrors; select one close to you.

  Next you are asked to select which packages to install.  The default
  settings in this dialog download a lot of packages that are not
  strictly necessary to run Unison with ssh.  If you don't want to
  install a package, click on it until ``skip'' is shown.  For a
  minimum installation, select only the packages ``cygwin'' and
  ``openssh,'' which come to about 1900KB; the full installation is
  much larger.

  \begin{quote} \em Note that you are plan to build unison using the free
    CygWin GNU C compiler, you need to install essential development
    packages such as ``gcc'', ``make'', ``fileutil'', etc; we refer to
    the file ``INSTALL.win32-cygwin-gnuc'' in the source distribution
    for further details.
  \end{quote}

  After the packages are downloaded and installed, the next dialog
  allows you to choose whether to ``Create Desktop Icon'' and ``Add to
  Start Menu.''  You make the call.
\item You can now delete the directory \verb|Foo| and its contents.
\end{enumerate}
Some people have reported problems using Cygwin's ssh with Unison.  If
you have trouble, you might try other ones instead:
\begin{verbatim}
  http://linuxmafia.com/ssh/win32.html
\end{verbatim}

\item You must set the environment variables HOME and PATH\@.
  Ssh will create a directory \verb|.ssh| in the directory given
  by HOME, so that it has a place to keep data like your public and
  private keys.  PATH must be set to include the Cygwin \verb|bin|
  directory, so that Unison can find the ssh executable.
  \begin{itemize}
  \item
    On Windows 95/98, add the lines
\begin{verbatim}
    set PATH=%PATH%;<SSHDIR>
    set HOME=<HOMEDIR>
\end{verbatim}
    to the file \verb|C:\AUTOEXEC.BAT|, where \verb|<HOMEDIR>| is the
    directory where you want ssh to create its \verb|.ssh| directory,
    and \verb|<SSHDIR>| is the directory where the executable
    \verb|ssh.exe| is stored; if you've installed Cygwin in the
    default location, this is \verb|C:\cygwin\bin|.  You will have to
    reboot your computer to take the changes into account.
  \item On Windows NT/2k/XP, open the environment variables dialog box:
    \begin{itemize}
    \item Windows NT: My Computer/Properties/Environment
    \item Windows 2k: My Computer/Properties/Advanced/Environment
      variables
    \end{itemize}
    then select Path and edit its value by appending \verb|;<SSHDIR>|
    to it, where \verb|<SSHDIR>| is the full name of the directory
    that includes the ssh executable; if you've installed Cygwin in
    the default location, this is \verb|C:\cygwin\bin|.
  \end{itemize}
  \item Test ssh from a DOS shell by typing
\begin{verbatim}
      ssh <remote host> -l <login name>
\end{verbatim}
    You should get a prompt for your password on \verb|<remote host>|,
    followed by a working connection.
  \item Note that \verb|ssh-keygen| may not work (fails with
  ``gethostname: no such file or directory'') on some systems.  This is
  OK: you can use ssh with your regular password for the remote
  system.
\item You should now be able to use Unison with an ssh connection. If
  you are logged in with a different user name on the local and remote
  hosts, provide your remote user name when providing the remote root
  (i.e., \verb|//username@host/path...|).
\end{enumerate}

\SECTION{Changes in Version \unisonversion}{news}{news}

\input{changes.tex}

\finishlater{
\SECTION{Other Synchronizers}{other}{other}

Unison is just one of several file synchronizers that are currently
available.

Check out:
  http://www.bell-labs.com/project/stage/
  I notice a bunch of people are also doing "data vaulting", e.g.,
    http://www.pc.ibm.com/us/thinkpad/datavault.html
  midnight commander??

Also:
  D. Duchamp
  A Toolkit Approach to Partially Disconnected Operation
  Proc. USENIX 1997 Ann. Technical Conf.
  USENIX, Anaheim CA, pp. 305-318, January 1997
}

\finishlater{
\SECTION{TODO}{todo}{ }

Things to write about:
\begin{itemize}
\item When started in 'socket server' mode, Unison prints 'server started' on
  stderr when it is ready to accept connections.
  (This may be useful for scripts that want to tell when a socket-mode server
  has finished initialization.)
\item {\tt DANGER.README}.
\end{itemize}
}

\finishlater{
Things to write about later:
\begin{itemize}
\item Document different reporting of file status when no archives
  were found.
\item Document buttons in graphical UI
\end{itemize}
}

\iftextversion
\SECTION{Junk}{ }{ }
\fi

\ifhevea\begin{rawhtml}</div>\end{rawhtml}\fi

\end{document}

      \end{quote}
    \fi
  \else
    \@opentoc{htoc}
    \tableofcontents
  \fi
}
\makeatother

\newcommand{\SNIP}[2]{%
\ifhevea\iftextversion
\begin{rawhtml}<pre>----SNIP----\end{rawhtml}
#1
#2 %
\begin{rawhtml}</pre>\end{rawhtml}%
\fi\fi
}

\newcommand{\sectionref}[2]{%
\ifhevea
  \iftextversion
    the section ``#2''
  \else
    the \url{##1}{#2} section%
  \fi
\else
  Section~\ref{#1} {[#2]}%
\fi
}

\newcommand{\bcpurl}[1]{\url{#1}}

\newcommand{\urlref}[2]{\bcpurl{##1}{#2}}
\newcommand{\ONEURL}[1]{%
  \iftextversion#1\else{\def~{\symbol{"7E}}\oneurl{#1}}\fi}
\newcommand{\URL}[2]{%
  \iftextversion#2 (#1)\else\bcpurl{#1}{#2}\fi}
\newcommand{\SHOWURL}[2]{%
  \ifhevea\URL{#1}{#2}\else#2\footnote{{\def~{\symbol{"7E}}\tt #1}}\fi}

% Usage: \SECTION{Title and menu item name}{tex label}{man section id}
\newcommand{\SECTION}[3]{%
  \ifhevea
    \SNIP{#1}{#3}%
    \iftextversion\else \@print{<hr>}\fi%
    \section*{\label{#2}#1}%
  \else
    \newpage
    \section{\label{#2}#1}%
    \addtocontents{htoc}{{\string\large\string\bf\string\urlref{#2}{#1}}\\}%
  \fi
}

\newcommand{\SUBSECTION}[2]{%
  \ifhevea
    \subsection*{\label{#2}#1}%
  \else
    \subsection{\label{#2}#1}%
    \addtocontents{htoc}{\hspace{10em}\bullet\string\urlref{#2}{#1}\\}
  \fi
}

\newcommand{\SUBSUBSECTION}[2]{%
  \ifhevea
    \subsubsection*{\label{#2}#1}%
  \else
    \subsubsection{\label{#2}#1}%
    \addtocontents{htoc}{\hspace{18em}\string\urlref{#2}{#1}\\}
  \fi
}

\newcommand{\TOPSUBSECTION}[2]{%
  \ifhevea\SNIP{#1}{#2}\fi
  \SUBSECTION{#1}{#2}%
}

% The quote-based macros looks a imperfect, perhaps due to the lack of
% alignment
% \newenvironment{textui}{{\em Textual Interface:}\begin{quote}}{\end{quote}}
% \newenvironment{tkui}{{\em Graphical Interface:}\begin{quote}}{\end{quote}}
\newenvironment{textui}{\medskip{\em Textual Interface:}\begin{itemize}\item[]
  }{\end{itemize}}
\newenvironment{tkui}{\medskip{\em Graphical Interface:}\begin{itemize}\item[]
  }{\end{itemize}}
\newenvironment{changesfromversion}[1]{%
  \noindent Changes since #1:
  \begin{itemize}
}{
  \end{itemize}
}

\newcommand{\incompatible}{%
  \iftextversion
    INCOMPATIBLE CHANGE:
  \else
    {\bf Incompatible change:}
  \fi}

\newcommand{\UNISONUSERS}{\URL{mailto:unison-users@yahoogroups.com}{{\tt
      unison-users@yahoogroups.com}}}
\newcommand{\UNISONHACKERS}{\URL{mailto:unison-hackers@lists.seas.upenn.edu}{{\tt
      unison-hackers@lists.seas.upenn.edu}}}

\ifhevea
 \makeatletter
 \let\oldmeta=\@meta
 \renewcommand{\@meta}{%
 \oldmeta
\ifdraft
 \begin{rawhtml}
 <META name="Author" content="Benjamin C. Pierce">
 <link rel="stylesheet" href="/home/bcpierce/pub/unison/unison.css">
 \end{rawhtml}
\else
 \begin{rawhtml}
 <META name="Author" content="Benjamin C. Pierce">
 <link rel="stylesheet" href="http://www.cis.upenn.edu/~bcpierce/unison/unison.css">
 \end{rawhtml}
\fi
}
 \makeatother
\fi

\fulltrue

%\newcommand{\NT}[1]{\(\langle\)\textit{#1}\(\rangle\)}
\newcommand{\NT}[1]{\textit{#1}}
\newcommand{\ARG}[1]{\texttt{\textit{#1}}}

%%%%%%%%%%%%%%%%%%%%%%%%%%%%%%%%%%%%%%%%%%%%%%%%%%%%%%%%%%%%%%%%%%%%%%
%%%%%%%%%%%%%%%%%%%%%%%%%%%%%%%%%%%%%%%%%%%%%%%%%%%%%%%%%%%%%%%%%%%%%%
\begin{document}

\ifhevea\begin{rawhtml}<div id="manualbody">\end{rawhtml}\fi

\ifhevea\else\bigskip\fi%
\ifdraft%
\begin{center}%
{\Huge \ifhevea\red\fi DraftDraftDraftDraft}%
\end{center}%
\ifhevea\else \bigskip \fi
\fi

\ifhevea\begin{rawhtml}<div id="manualheader">\end{rawhtml}%
\else \thispagestyle{empty}
\fi%
\SNIP{About Unison}{about}%
\iftextversion
  \section*{Unison File Synchronizer
%%   \\
%%   \ONEURL{http://www.cis.upenn.edu/\home{bcpierce}/unison}
  \\
  Version
  \unisonversion
  }
\else%
  \ifhevea\else \vspace*{2in} \fi%
  \begin{center}%
  \Huge{\ifhevea\black\else\bf \fi Unison File  Synchronizer}%
%%  \ifhevea \\ \else \\[2ex] \fi
%%   \large
%%   \ONEURL{http://www.cis.upenn.edu/\home{bcpierce}/unison}
  \ifhevea \\ \else \\[2ex] \fi%
  \huge {\ifhevea\black\else\bf \fi User Manual and Reference Guide}%
  \ifhevea \\ \else \\[6ex] \fi%
  \LARGE%
  Version \unisonversion \\[4ex] %
  % \today %
  \large Copyright 1998-2015, 2017, Benjamin C. Pierce
  \end{center}%
\fi%
%
%
\ifhevea\begin{rawhtml}</div>\end{rawhtml}\fi

\ifhevea\else\newpage\fi
\TABLEOFCONTENTS
\ifhevea\else\newpage\fi

\SECTION{Overview}{overview}{ }

\input{short}

\ifhevea\else\bigskip\fi

% \begin{quote}
% {\bf\ifhevea\red\fi Warning:} The current implementation of Unison is
% considered beta-test software.  It is in daily use by quite a few
% people, but there are still undoubtedly some bugs.  If you choose to
% use it to synchronize important data, please pay careful attention
% to what it is doing!  Also, the installation/setup procedure is not
% yet as smooth as we want it to be.
% \end{quote}


\SECTION{Preface}{intro}{ }

\TOPSUBSECTION{People}{people}

\URL{http://www.cis.upenn.edu/\home{bcpierce}/}{Benjamin Pierce} leads the
Unison project.
%
The current version of Unison was designed and implemented by
    \URL{http://www.research.att.com/\home{trevor}/}{Trevor Jim},
    \URL{http://www.cis.upenn.edu/\home{bcpierce}/}{Benjamin Pierce},
and
    \URL{http://www.pps.jussieu.fr/\home{vouillon}/}{J\'{e}r\^{o}me Vouillon},
with
    \URL{http://alan.petitepomme.net/}{Alan Schmitt},
    {Malo Denielou},
    \URL{http://www.brics.dk/\home{zheyang}/}{Zhe Yang},
    Sylvain Gommier, and
    Matthieu Goulay.
%
The Mac user interface was started by Trevor Jim and enormously improved by
Ben Willmore.
%
Our implementation of the
  \URL{http://samba.org/rsync/}{rsync}
  protocol was built by
  \URL{http://www.eecs.harvard.edu/\home{nr}/}{Norman Ramsey}
  and Sylvain Gommier.  It is based on
  \URL{http://samba.anu.edu.au/\home{tridge}/}{Andrew Tridgell}'s
  \URL{http://samba.anu.edu.au/\home{tridge}/phd\_thesis.pdf}{thesis work}
  and inspired by his
  \URL{http://samba.org/rsync/}{rsync}
  utility.
% \finish{Our low-level fingerprinting implementation uses an algorithm
% by Michael Rabin and incorporates some coding tricks from Andrei
% Broder and Mike Burrows.}
%
The mirroring and merging functionality was implemented by
  Sylvain Roy, improved by Malo Denielou, and improved yet further by
  St\'ephane Lescuyer.
%
 \URL{http://wwwfun.kurims.kyoto-u.ac.jp/\home{garrigue}/}{Jacques Garrigue}
 contributed the original Gtk version of the user
  interface; the Gtk2 version was built by Stephen Tse.
%
Sundar Balasubramaniam helped build a prototype implementation of
an earlier synchronizer in Java.
\URL{http://www.cis.upenn.edu/\home{ishin}/}{Insik Shin}
and
\URL{http://www.cis.upenn.edu/\home{lee}/}{Insup Lee} contributed design
ideas to this implementation.
\URL{http://research.microsoft.com/\home{fournet}/}{Cedric Fournet}
contributed to an even earlier prototype.

\TOPSUBSECTION{Mailing Lists and Bug Reporting}{lists}

\input{contactsbody}

\TOPSUBSECTION{Development Status}{status}

Unison is no longer under active development as a research
project.  (Our research efforts are now focused on a follow-on
project called Boomerang, described at
\ONEURL{http://www.cis.upenn.edu/\home{bcpierce}/harmony}.)
At this point, there is no one whose job it is to maintain Unison,
fix bugs, or answer questions.

However, the original developers are all still using Unison daily.  It
will continue to be maintained and supported for the foreseeable future,
and we will occasionally release new versions with bug fixes, small
improvements, and contributed patches.

Reports of bugs affecting correctness or safety are of interest to many
people and will generally get high priority.  Other bug reports will be
looked at as time permits.  Bugs should be reported to the users list at
\UNISONUSERS.

Feature requests are welcome, but will probably just be added to the
ever-growing todo list.  They should also be sent to \UNISONUSERS.

Patches are even more welcome.  They should be sent to
\UNISONHACKERS.
(Since safety and robustness are Unison's most important properties,
patches will be held to high standards of clear design and clean coding.)
If you want to contribute to Unison, start by downloading the developer
tarball from the download page.  For some details on how the code is
organized, etc., see the file {\tt CONTRIB}.

\TOPSUBSECTION{Copying}{copying}

This file is part of Unison.

    Unison is free software: you can redistribute it and/or modify
    it under the terms of the GNU General Public License as published by
    the Free Software Foundation, either version 3 of the License, or
    (at your option) any later version.

    Unison is distributed in the hope that it will be useful,
    but WITHOUT ANY WARRANTY; without even the implied warranty of
    MERCHANTABILITY or FITNESS FOR A PARTICULAR PURPOSE.  See the
    GNU General Public License for more details.

    The GNU Public License can be found at
    \ONEURL{http://www.gnu.org/licenses}.  A copy is also included in the
    Unison source distribution in the file {\tt COPYING}.

\TOPSUBSECTION{Acknowledgements}{ack}

Work on Unison has been supported by the National Science Foundation
under grants CCR-9701826 and ITR-0113226, {\em Principles and Practice of
  Synchronization}, and by University of Pennsylvania's Institute for
Research in Cognitive Science (IRCS).

\SECTION{Installation}{install}{install}

Unison is designed to be easy to install.  The following sequence of
steps should get you a fully working installation in a few minutes.  If
you run into trouble, you may find the suggestions on the
\SHOWURL{http://www.cis.upenn.edu/\home{bcpierce}/unison/faq.html}{Frequently Asked
Questions page} helpful.  Pre-built binaries are available for a
variety of platforms.

Unison can be used with either of two user interfaces:
\begin{enumerate}
\item a simple textual interface, suitable for dumb terminals (and
running from scripts), and
\item a more sophisticated graphical interface, based on Gtk2 (on
       Linux/Windows) or the native UI framework (on OSX).
\end{enumerate}

You will need to install a copy of Unison on every machine that you
want to synchronize.  However, you only need the version with a
graphical user interface (if you want a GUI at all) on the machine
where you're actually going to display the interface (the \CLIENT{}
machine).  Other machines that you synchronize with can get along just
fine with the textual version.


\SUBSECTION{Downloading Unison}{download}

The Unison download site lives under
\ONEURL{http://www.cis.upenn.edu/\home{bcpierce}/unison}.

If a pre-built binary of Unison is available for the client machine's
architecture, just download it and put it somewhere in your search
path (if you're going to invoke it from the command line) or on your
desktop (if you'll be click-starting it).

The executable file for the graphical version (with a name including
\verb|gtkui|) actually provides {\em both} interfaces: the graphical one
appears by default, while the textual interface can be selected by including
\verb|-ui text| on the command line.  The \verb|textui| executable
provides just the textual interface.

If you don't see a pre-built executable for your architecture, you'll
need to build it yourself.  See \sectionref{building}{Building Unison from Scratch}.
There are also a small number of contributed ports to other
architectures that are not maintained by us.  See the
\SHOWURL{http://www.cis.upenn.edu/\home{bcpierce}/unison/download.html}{Contributed
Ports page} to check what's available.

Check to make sure that what you have downloaded is really executable.
Either click-start it, or type \showtt{unison -version} at the command
line.

Unison can be used in three different modes: with different directories on a
single machine, with a remote machine over a direct socket connection, or
with a remote machine using {\tt ssh} for authentication and secure
transfer.  If you intend to use the last option, you may need to install
{\tt ssh}; see \sectionref{ssh}{Installing Ssh}.

\SUBSECTION{Running Unison}{afterinstall}

Once you've got Unison installed on at least one system, read
\sectionref{tutorial}{Tutorial} of the user manual (or type \showtt{unison -doc
  tutorial}) for instructions on how to get started.


\SUBSECTION{Upgrading}{upgrading}

Upgrading to a new version of Unison is as simple as throwing away the old
binary and installing the new one.

Before upgrading, it is a good idea to run the {\em old} version one last
time, to make sure all your replicas are completely synchronized.  A new
version of Unison will sometimes introduce a different format for the
archive files used to remember information about the previous state of the
replicas.  In this case, the old archive will be ignored (not deleted --- if
you roll back to the previous version of Unison, you will find the old
archives intact), which means that any differences between the replicas will
show up as conflicts that need to be resolved manually.


\SUBSECTION{Building Unison from Scratch}{building}

If a pre-built image is not available, you will need to compile it from
scratch; the sources are available from the same place as the binaries.

In principle, Unison should work on any platform to which OCaml has been
ported and on which the \verb|Unix| module is fully implemented.  It has
been tested on many flavors of Windows (98, NT, 2000, XP) and Unix (OS X,
Solaris, Linux, FreeBSD), and on both 32- and 64-bit architectures.


\SUBSUBSECTION{Unix}{build-unix}

Unison can be built with or without a graphical user interface (GUI). The
build system will decide automatically depending on the libraries installed
on your system, but you can also type {\tt make UISTYLE=text} to build
Unison without GUI.

You'll need the Objective Caml compiler,
available from \ONEURL{http://caml.inria.fr}.  OCaml is available from most
package managers
Building and installing OCaml
on Unix systems is very straightforward; just follow the instructions in the
distribution.  You'll probably want to build the native-code compiler in
addition to the bytecode compiler, as Unison runs much faster when compiled
to native code, but this is not absolutely necessary.
%
(Quick start: on many systems, the following sequence of commands will
get you a working and installed compiler: first do {\tt make world opt},
then {\tt su} to root and do {\tt make install}.)

You'll also need the GNU {\tt make} utility, which is standard on most Unix
systems.  Unison's build system is not parallelizable, so don't use flags
that cause it to start processes in parallel (e.g. -j).

Once you've got OCaml installed, grab a copy of the Unison sources, unzip
and untar them, change to the new \showtt{unison} directory, and type ``{\tt
  make UISTYLE=text}''.  The result should be an executable file called
\showtt{unison}.  Type \showtt{./unison} to make sure the program is
executable.  You should get back a usage message.

If you want to build the graphical user interface, you will need to install
some additional things:
\begin{itemize}
\item The Gtk2 development libraries (package {\tt libgtk2.0-dev} on debian
based systems).
\item OCaml bindings for Gtk2. Install them from your software repositories
(package {\tt liblablgtk2-ocaml} on debian based systems). Also available
from \ONEURL{http://wwwfun.kurims.kyoto-u.ac.jp/soft/olabl/lablgtk.html}.
\item Pango, a text rendering library and a part of Gtk2. On some systems
(e.g. Ubuntu) the bindings between Pango and OCaml need to be installed
explicitly (package {\tt liblablgtk-extras-ocaml-dev} on Ubuntu).
\end{itemize}
Type {\tt make src} to build Unison. If Gtk2 is available on the system,
Unison with a GUI will be built automatically.

Put the \verb|unison| executable somewhere in your search path, either by
adding the Unison directory to your PATH variable or by copying the
executable to some standard directory where executables are stored.  Or just
type {\tt make install} to install Unison to {\tt \$HOME/bin/unison}.

\SUBSUBSECTION{Mac OS X}{build-osx}

To build the text-only user interface, follow the instructions above for
building on Unix systems.  You should do this first, even if you are also
planning on building the GUI, just to make sure it works.

To build the basic GUI version, you'll first need to download and install
the XCode developer tools from Apple.  Once this is done, just type {\tt
  make} in the {\tt src} directory, and if things go well you
should get an application that you can move from {\tt
  uimac14/build/Default/Unison.app} to wherever you want it.

\SUBSUBSECTION{Windows}{build-win}

Although the binary distribution should work on any version of Windows,
some people may want to build Unison from scratch on those systems too.

\paragraph{Bytecode version:} The simpler but slower compilation option
to build a Unison executable is to build a bytecode version.  You need
first install Windows version of the OCaml compiler (version 3.07 or
later, available from \ONEURL{http://caml.inria.fr}).  Then grab a copy
of Unison sources and type
\begin{verbatim}
       make NATIVE=false
\end{verbatim}
to compile the bytecode.  The result should be an executable file called
\verb|unison.exe|.

\paragraph{Native version:} Building a more efficient, native version of
Unison on Windows requires a little more work.  See the file {\tt
  INSTALL.win32} in the source code distribution.


\SUBSUBSECTION{Installation Options}{build-opts}

The \verb|Makefile| in the distribution includes several switches that
can be used to control how Unison is built.  Here are the most useful
ones:
\begin{itemize}
\item Building with \verb|NATIVE=true| uses the native-code OCaml
compiler, yielding an executable that will run quite a bit faster. We use
this for building distribution versions.
\item Building with \verb|make DEBUGGING=true| generates debugging
symbols.
\item Building with \verb|make STATIC=true| generates a (mostly)
statically linked executable.  We use this for building distribution
versions, for portability.
\end{itemize}
%\finish{Any other important ones?}


\SECTION{Tutorial}{tutorial}{tutorial}

%\finish{Put a pointer somewhere in here to the typical profile in the
%  reference section.}

\SUBSECTION{Preliminaries}{prelim}

Unison can be used with either of two user interfaces:
\begin{enumerate}
\item a straightforward textual interface and
\item a more sophisticated graphical interface
\end{enumerate}
The textual interface is more convenient for running from scripts and
works on dumb terminals; the graphical interface is better for most
interactive use.  For this tutorial, you can use either.  If you are running
Unison from the command line, just typing {\tt unison}
will select either the text or the graphical interface, depending on which
has been selected as default when the executable you are running was
built.  You can force the text interface even if graphical is the default by
adding {\tt -ui text}.
The other command-line arguments to both versions are identical.

The graphical version can also be run directly by clicking on its icon, but
this may require a little set-up (see \sectionref{click}{Click-starting
  Unison}).  For this tutorial, we assume that you're starting it from the
command line.

Unison can synchronize files and directories on a single machine, or
between two machines on a network.  (The same program runs on both
machines; the only difference is which one is responsible for
displaying the user interface.)  If you're only interested in a
single-machine setup, then let's call that machine the \CLIENT{}.  If
you're synchronizing two machines, let's call them \CLIENT{} and
\SERVER.

\SUBSECTION{Local Usage}{local}

Let's get the client machine set up first and see how to synchronize
two directories on a single machine.

Follow the instructions in \sectionref{install}{Installation} to either
download or build an executable version of Unison, and install it
somewhere on your search path.  (If you just want to use the textual user
interface, download the appropriate textui binary.  If you just want to
the graphical interface---or if you will use both interfaces [the gtkui
binary actually has both compiled in]---then download the gtkui binary.)

Create a small test directory {\tt a.tmp} containing a couple of files
and/or subdirectories, e.g.,
\begin{verbatim}
       mkdir a.tmp
       touch a.tmp/a a.tmp/b
       mkdir a.tmp/d
       touch a.tmp/d/f
\end{verbatim}
Copy this directory to b.tmp:
\begin{verbatim}
       cp -r a.tmp b.tmp
\end{verbatim}

Now try synchronizing {\tt a.tmp} and {\tt b.tmp}.  (Since they are
identical, synchronizing them won't propagate any changes, but Unison
will remember the current state of both directories so that it will be
able to tell next time what has changed.)  Type:
\begin{verbatim}
       unison a.tmp b.tmp
\end{verbatim}
(You may need to add \verb|-ui text|, depending how your unison binary was built.)

\begin{textui}
You should see a message notifying you that all the files are actually
equal and then get returned to the command line.
\end{textui}

\begin{tkui}
You should get a big empty window with a message at the bottom
notifying you that all files are identical.  Choose the Exit item from
the File menu to get back to the command line.
\end{tkui}

Next, make some changes in a.tmp and/or b.tmp.  For example:
\begin{verbatim}
        rm a.tmp/a
        echo "Hello" > a.tmp/b
        echo "Hello" > b.tmp/b
        date > b.tmp/c
        echo "Hi there" > a.tmp/d/h
        echo "Hello there" > b.tmp/d/h
\end{verbatim}
Run Unison again:
\begin{verbatim}
       unison a.tmp b.tmp
\end{verbatim}

This time, the user interface will display only the files that have
changed.  If a file has been modified in just one
replica, then it will be displayed with an arrow indicating the
direction that the change needs to be propagated.  For example,
\begin{verbatim}
                 <---  new file   c  [f]
\end{verbatim}
\noindent
indicates that the file {\tt c} has been modified only in the second
replica, and that the default action is therefore to propagate the new
version to the first replica.  To {\bf f}ollow Unison's recommendation,
press the ``f'' at the prompt.

If both replicas are modified and their contents are different, then
the changes are in conflict: \texttt{<-?->} is displayed to indicate
that Unison needs guidance on which replica should override the
other.
\begin{verbatim}
     new file  <-?->  new file   d/h  []
\end{verbatim}
By default, neither version will be propagated and both
replicas will remain as they are.

If both replicas have been modified but their new contents are the same
(as with the file {\tt b}), then no propagation is necessary and
nothing is shown.  Unison simply notes that the file is up to date.

These display conventions are used by both versions of the user
interface.  The only difference lies in the way in which Unison's
default actions are either accepted or overridden by the user.

\begin{textui}
The status of each modified file is displayed, in turn.
When the copies of a file in the two replicas are not identical, the
user interface will ask for instructions as to how to propagate the
change.  If some default action is indicated (by an arrow), you can
simply press Return to go on to the next changed file.  If you want to
do something different with this file, press ``\verb|<|'' or ``\verb|>|'' to force
the change to be propagated from right to left or from left to right,
or else press ``\verb|/|'' to skip this file and leave both replicas alone.
When it reaches the end of the list of modified files, Unison will ask
you one more time whether it should proceed with the updates that have
been selected.

When Unison stops to wait for input from the user, pressing ``\verb|?|''
will always give a list of possible responses and their meanings.
\end{textui}

\begin{tkui}
The main window shows all the files that have been modified in either
{\tt a.tmp} or {\tt b.tmp}.  To override a default action (or to select
an action in the case when there is no default), first select the file, either
by clicking on its name or by using the up- and down-arrow keys.  Then
press either the left-arrow or ``\verb|<|'' key (to cause the version in b.tmp to
propagate to a.tmp) or the right-arrow or ``\verb|>|'' key (which makes the a.tmp
version override b.tmp).

Every keyboard command can also be invoked from the menus at the top
of the user interface.  (Conversely, each menu item is annotated with
its keyboard equivalent, if it has one.)

When you are satisfied with the directions for the propagation of changes
as shown in the main window, click the ``Go'' button to set them in
motion.  A check sign will be displayed next to each filename
when the file has been dealt with.
\end{tkui}


\SUBSECTION{Remote Usage}{remote}

Next, we'll get Unison set up to synchronize replicas on two different
machines.

Follow the instructions in the Installation section to download or
build an executable version of Unison on the server machine, and
install it somewhere on your search path.  (It doesn't matter whether
you install the textual or graphical version, since the copy of Unison on
the server doesn't need to display any user interface at all.)

It is important that the version of Unison installed on the server
machine is the same as the version of Unison on the client machine.
But some flexibility on the version of Unison at the client side can
be achieved by using the \verb|-addversionno| option; see
\sectionref{prefs}{Preferences}.

Now there is a decision to be made.  Unison provides two methods for
communicating between the client and the server:
\begin{itemize}
\item {\em Remote shell method}: To use this method, you must have
  some way of invoking remote commands on the server from the client's
  command line, using a facility such as \verb|ssh|.
  This method is more convenient (since there is no need to manually
  start a ``unison server'' process on the server) and also more
  secure (especially if you use \verb|ssh|).

\item {\em Socket method}: This method requires only that you can get
  TCP packets from the client to the server and back.  A draconian
  firewall can prevent this, but otherwise it should work anywhere.
\end{itemize}

Decide which of these you want to try, and continue with
\sectionref{rshmeth}{Remote Shell Method} or
\sectionref{socketmeth}{Socket Method}, as appropriate.


\SUBSECTION{Remote Shell Method}{rshmeth}

The standard remote shell facility on Unix systems is \verb|ssh|, which provides the
same functionality as the older \verb|rsh| but much better security.  Ssh is available from
\ONEURL{http://www.openssh.org}.  See section~\ref{ssh-win}
for installation instructions for the Windows version.

Running
\verb|ssh| requires some coordination between the client and server
machines to establish that the client is allowed to invoke commands on
the server; please refer to the \verb|ssh| documentation
for information on how to set this up.  The examples in this section
use \verb|ssh|, but you can substitute \verb|rsh| for \verb|ssh| if
you wish.

First, test that we can invoke Unison on the server from the client.
Typing
\begin{alltt}
        ssh \NT{remotehostname} unison -version
\end{alltt}
should print the same version information as running
\begin{verbatim}
        unison -version
\end{verbatim}
locally on the client.  If remote execution fails, then either
something is wrong with your ssh setup (e.g., ``permission denied'')
or else the search path that's being used when executing commands on
the server doesn't contain the \verb|unison| executable (e.g.,
``command not found'').

Create a test directory {\tt a.tmp} in your home directory on the client
machine.

Test that the local unison client can start and connect to the
remote server.  Type
\begin{alltt}
          unison -testServer a.tmp ssh://\NT{remotehostname}/a.tmp
\end{alltt}

Now cd to your home directory and type:
\begin{verbatim}
          unison a.tmp ssh://remotehostname/a.tmp
\end{verbatim}
The result should be that the entire directory {\tt a.tmp} is propagated
from the client to your home directory on the server.

After finishing the first synchronization, change a few files and try
synchronizing again.  You should see similar results as in the local
case.

If your user name on the server is not the same as on the client, you
need to specify it on the command line:
\begin{verbatim}
          unison a.tmp ssh://username@remotehostname/a.tmp
\end{verbatim}

\noindent {\it Notes:}
\begin{itemize}
\item If you want to put \verb|a.tmp| some place other than your home
directory on the remote host, you can give an absolute path for it by
adding an extra slash between \verb|remotehostname| and the beginning
of the path:
\begin{verbatim}
          unison a.tmp ssh://remotehostname//absolute/path/to/a.tmp
\end{verbatim}

\item You can give an explicit path for the \verb|unison| executable
  on the server by using the command-line option \showtt{-servercmd
    /full/path/name/of/unison} or adding
  \showtt{servercmd=/full/path/name/of/unison} to your profile (see
  \sectionref{profile}{Profiles}).  Similarly, you can specify a
  explicit path for the \verb|ssh| program using the \showtt{-sshcmd}
  option.
  Extra arguments can be passed to \verb|ssh| by setting the
  \verb|-sshargs| preference.
\end{itemize}


\SUBSECTION{Socket Method}{socketmeth}

\begin{quote}
  {\bf\ifhevea\red\fi Warning:} The socket method is
  insecure: not only are the texts of your changes transmitted over
  the network in unprotected form, it is also possible for anyone in
  the world to connect to the server process and read out the contents
  of your filesystem!  (Of course, to do this they must understand the
  protocol that Unison uses to communicate between client and server,
  but all they need for this is a copy of the Unison sources.)  The socket
  method is provided only for expert users with specific needs; everyone
  else should use the \verb|ssh| method.
\end{quote}

To run Unison over a socket connection, you must start a Unison
daemon process on the server.  This process runs continuously,
waiting for connections over a given socket from client machines
running Unison and processing their requests in turn.

To start the daemon, type
\begin{verbatim}
       unison -socket NNNN
\end{verbatim}
on the server machine, where {\tt NNNN} is the socket number that the
daemon should listen on for connections from clients.  ({\tt NNNN} can
be any large number that is not being used by some other program; if
\texttt{NNNN} is already in use, Unison will exit with an error
message.)  Note that paths specified by the client will be interpreted
relative to the directory in which you start the server process; this
behavior is different from the ssh case, where the path is relative to
your home directory on the server.

Create a test directory {\tt a.tmp} in your home directory on the
client machine.  Now type:
\begin{alltt}
       unison a.tmp socket://\NT{remotehostname}:NNNN/a.tmp
\end{alltt}
The result should be that the entire directory {\tt a.tmp} is
propagated from the client to the server (\texttt{a.tmp} will be
created on the server in the directory that the server was started
from).
%
After finishing the first synchronization, change a few files and try
synchronizing again.  You should see similar results as in the local
case.

Since the socket method is not used by many people, its functionality is
rather limited.  For example, the server can only deal with one client at a
time.


\SUBSECTION{Using Unison for All Your Files}{usingit}

Once you are comfortable with the basic operation of Unison, you may
find yourself wanting to use it regularly to synchronize your commonly
used files.  There are several possible ways of going about this:

\begin{enumerate}
\item Synchronize your whole home directory, using the Ignore facility
(see \sectionref{ignore}{Ignoring Paths})
to avoid synchronizing temporary files and things that only belong on
one host.
\item Create a subdirectory called {\tt shared} (or {\tt current}, or
whatever) in your home directory on each host, and put all the files
you want to synchronize into this directory.
\item Create a subdirectory called {\tt shared} (or {\tt current}, or
whatever) in your home directory on each host, and put {\em links to}
all the files you want to synchronize into this directory.  Use the
{\tt follow} preference (see \sectionref{symlinks}{Symbolic Links}) to make
Unison treat these links as transparent.
\item Make your home directory the root of the synchronization, but
tell Unison to synchronize only some of the files and subdirectories
within it on any given run.  This can be accomplished by using the {\tt -path} switch
on the command line:
\begin{alltt}
       unison /home/\NT{username} ssh://\NT{remotehost}//home/\NT{username} -path shared
\end{alltt}
The {\tt -path} option can be used as many times as needed, to
synchronize several files or subdirectories:
\begin{alltt}
       unison /home/\NT{username} ssh://\NT{remotehost}//home/\NT{username} \verb|\|
          -path shared \verb|\|
          -path pub \verb|\|
          -path .netscape/bookmarks.html
\end{alltt}
These \verb|-path| arguments can also be put in your preference file.
See \sectionref{prefs}{Preferences} for an example.
\end{enumerate}

Most people find that they only need to maintain a profile (or
profiles) on one of the hosts that they synchronize, since Unison is
always initiated from this host.  (For example, if you're
synchronizing a laptop with a fileserver, you'll probably always run
Unison on the laptop.)  This is a bit different from the usual
situation with asymmetric mirroring programs like \verb|rdist|, where
the mirroring operation typically needs to be initiated from the
machine with the most recent changes.  \sectionref{profile}{Profiles}
covers the syntax of Unison profiles, together with some sample profiles.

Some tips on improving Unison's performance can be found on the
\SHOWURL{http://www.cis.upenn.edu/\home{bcpierce}/unison/faq.html}{Frequently
  Asked Questions page}.

\SUBSECTION{Using Unison to Synchronize More Than Two Machines}{usingmultiple}

Unison is designed for synchronizing pairs of replicas.  However, it is
possible to use it to keep larger groups of machines in sync by performing
multiple pairwise synchronizations.

If you need to do this, the most reliable way to set things up is to
organize the machines into a ``star topology,'' with one machine designated
as the ``hub'' and the rest as ``spokes,'' and with each spoke machine
synchronizing only with the hub.  The big advantage of the star topology is
that it eliminates the possibility of confusing ``spurious conflicts''
arising from the fact that a separate archive is maintained by Unison for
every pair of hosts that it synchronizes.


\SUBSECTION{Going Further}{further}

On-line documentation for the various features of Unison
can be obtained either by typing
\begin{verbatim}
        unison -doc topics
\end{verbatim}
\noindent
at the command line, or by selecting the Help menu in the graphical
user interface.
\iftextversion
The same information is also available in a typeset User's
Manual (HTML or PostScript format) through
\ONEURL{http://www.cis.upenn.edu/\home{bcpierce}/unison}.
\else
The on-line information and the printed manual are essentially identical.
\fi

If you use Unison regularly, you should subscribe to one of the mailing
lists, to receive announcements of new versions.  See
\sectionref{lists}{Mailing Lists and Bug Reporting}.

\SECTION{Basic Concepts}{basics}{basics}

To understand how Unison works, it is necessary to discuss a few
straightforward concepts.
%
These concepts are developed more rigorously and at more length in a number
of papers, available at \ONEURL{http://www.cis.upenn.edu/\home{bcpierce}/papers}.
But the informal presentation here should be enough for most users.


\SUBSECTION{Roots}{roots}

A replica's {\em root} tells Unison where to find a set of files to be
synchronized, either on the local machine or on a remote host.
For example,
\begin{alltt}
      \NT{relative/path/of/root}
\end{alltt}
\noindent
specifies a local root relative to the directory where Unison is
started, while
\begin{alltt}
      /\NT{absolute/path/of/root}
\end{alltt}
\noindent
specifies a root relative to the top of the local filesystem,
independent of where Unison is running.  Remote roots can begin with
\verb|ssh://|,
\verb|rsh://|
to indicate that the remote server should be started with rsh or ssh:
\begin{alltt}
      ssh://\NT{remotehost}//\NT{absolute/path/of/root}
      rsh://\NT{user}@\NT{remotehost}/\NT{relative/path/of/root}
\end{alltt}
If the remote server is already running (in the socket mode), then the syntax
\begin{alltt}
      socket://\NT{remotehost}:\NT{portnum}//\NT{absolute/path/of/root}
      socket://\NT{remotehost}:\NT{portnum}/\NT{relative/path/of/root}
\end{alltt}
\noindent
is used to specify the hostname and the port that the client Unison should
use to contact it.

The syntax for roots is based on that of URIs (described in RFC 2396).
The full grammar is:
\begin{alltt}
  \NT{replica} ::= [\NT{protocol}:]//[\NT{user}@][\NT{host}][:\NT{port}][/\NT{path}]
           |  \NT{path}

  \NT{protocol} ::= file
            |  socket
            |  ssh
            |  rsh

  \NT{user} ::= [-_a-zA-Z0-9]+

  \NT{host} ::= [-_a-zA-Z0-9.]+

  \NT{port} ::= [0-9]+
\end{alltt}
When \verb|path| is given without any protocol prefix, the protocol is
assumed to be \verb|file:|.  Under Windows, it is possible to
synchronize with a remote directory using the \verb|file:| protocol over
the Windows Network Neighborhood.  For example,
\begin{verbatim}
       unison foo //host/drive/bar
\end{verbatim}
\noindent
synchronizes the local directory \verb|foo| with the directory
\verb|drive:\bar| on the machine \verb|host|, provided that \verb|host|
is accessible via Network Neighborhood.  When the \verb|file:| protocol
is used in this way, there is no need for a Unison server to be running
on the remote host.  However, running Unison this way is only a good
idea if the remote host is reached by a very fast network connection,
since the full contents of every file in the remote replica will have to
be transferred to the local machine to detect updates.

The names of roots are {\em canonized} by Unison before it uses them
to compute the names of the corresponding archive files, so {\tt
  //saul//home/bcpierce/common} and {\tt //saul.cis.upenn.edu/common}
will be recognized as the same replica under different names.

\SUBSECTION{Paths}{paths}

A {\em path} refers to a point {\em within} a set of files being
synchronized; it is specified relative to the root of the replica.

Formally, a path is just a sequence of names, separated by \verb|/|.
Note that the path separator character is always a forward slash, no
matter what operating system Unison is running on.  Forward slashes
are converted to backslashes as necessary when paths are converted to
filenames in the local filesystem on a particular host.
%
(For example, suppose that we run Unison on a Windows system, synchronizing
the local root \verb|c:\pierce| with the root
\verb|ssh://saul.cis.upenn.edu/home/bcpierce| on a Unix server.  Then
the path \verb|current/todo.txt| refers to the file
\verb|c:\pierce\current\todo.txt| on the client and
\verb|/home/bcpierce/current/todo.txt| on the server.)

The empty path (i.e., the empty sequence of names) denotes the whole
replica.  Unison displays the empty path as ``\verb|[root]|.''

If \verb|p| is a path and \verb|q| is a path beginning with \verb|p|, then
\verb|q| is said to be a {\em descendant} of \verb|p|.  (Each path is also a
descendant of itself.)


\SUBSECTION{What is an Update?}{updates}

The {\em contents} of a path \verb|p| in a particular replica could be a
file, a directory, a symbolic link, or absent (if \verb|p| does not
refer to anything at all in that replica).  More specifically:
\begin{itemize}
\item If \verb|p| refers to an ordinary file, then the
contents of \verb|p| are the actual contents of this file (a string of bytes)
plus the current permission bits of the file.
\item If \verb|p| refers to a symbolic link, then the contents of \verb|p|
are just the string specifying where the link points.
\item If \verb|p| refers to a directory, then the
contents of \verb|p| are just the token ``DIRECTORY'' plus the current
permission bits of the directory.
\item If \verb|p| does not refer to anything in this replica, then the
contents of \verb|p| are the token ``ABSENT.''
\end{itemize}
Unison keeps a record of the contents of each path after each
successful synchronization of that path (i.e., it remembers the
contents at the last moment when they were the same in the two
replicas).

We say that a path is {\em updated} (in some replica) if its current
contents are different from its contents the last time it was successfully
synchronized.  Note that whether a path is updated has nothing to do with
its last modification time---Unison considers only the contents when
determining whether an update has occurred.  This means that touching a file
without changing its contents will {\em not} be recognized as an update.  A
file can even be changed several times and then changed back to its original
contents; as long as Unison is only run at the end of this process, no
update will be recognized.

What Unison actually calculates is a close approximation to this
definition; see \sectionref{caveats}{Caveats and Shortcomings}.

\SUBSECTION{What is a Conflict?}{conflicts}

A path is said to be {\em conflicting} if the following conditions all hold:
\begin{enumerate}
\item it has been updated in one replica,
\item it or any of its descendants has been updated in the other
  replica,
and
\item its contents in the two replicas are not identical.
\end{enumerate}

\finishlater{Note that this isn't precisely what we implement, in the
  case of directory permission changes!}


\SUBSECTION{Reconciliation}{recon}

Unison operates in several distinct stages:
\begin{enumerate}
\item On each host, it compares its archive file (which records
the state of each path in the replica when it was last synchronized)
with the current contents of the replica, to determine which paths
have been updated.
\item It checks for ``false conflicts'' --- paths that have been
updated on both replicas, but whose current values are identical.
These paths are silently marked as synchronized in the archive files
in both replicas.
\item It displays all the updated paths to the user.  For updates that
do not conflict, it suggests a default action (propagating the new
contents from the updated replica to the other).  Conflicting updates
are just displayed.  The user is given an opportunity to examine the
current state of affairs, change the default actions for
nonconflicting updates, and choose actions for conflicting updates.
\item It performs the selected actions, one at a time.  Each action is
performed by first transferring the new contents to a temporary file
on the receiving host, then atomically moving them into place.
\item It updates its archive files to reflect the new state of the
replicas.
\end{enumerate}

\TOPSUBSECTION{Invariants}{failures}

Given the importance and delicacy of the job that it performs, it is
important to understand both what a synchronizer does under normal
conditions and what can happen under unusual conditions such as system
crashes and communication failures.

% Unison deals with two sorts of information: the two replicas
% themselves and its own memory of the ``last synchronized state'' of
% each path in the replicas.  The latter is what allows it to detect
% correctly which replica is new when a file been updated.  Roughly,
% the sequence of actions that occur when Unison runs is:
% \begin{enumerate}
% \item It reads a private archive file stored with each replica
% and checks which paths on each replica have been updated.
% Technically, a path has been updated if its contents in a replica are
% different from the contents of that replica at the end of the last
% synchronization in which that path was successfully synchronized ---
% i.e., the last time the two replicas were equal at that path at the
% end of a run of Unison.  The ``contents'' of a path can be either a
% file, a directory, or nothing at all, so deleting a file or changing a
% directory to a file count as updates to the contents at that path.

% For efficiency, Unison does not try to calculate the set of updated
% paths exactly: it will sometimes falsely detect a change in a path
% whose contents have actually not changed (this can happen, for
% example, when the file's modification time has been changed, for some
% reason).  As long as this path has not been modified in the other
% replica, this ``conservativity'' in update detection is invisible to
% the user.  If the other replica {\em has} been modified, however, a
% ``false conflict'' may be reported.

% \item It combines the lists of paths that (may) have been updated in
% the two replicas, assigns default actions to those where the change
% was in one replica only, and records a conflict for those that were
% changed in both replicas.

% \item The current contents of the paths on this list are then
% compared, to see if they actually differ.  (This is done by comparing
% fingerprints, not transferring the whole files.)  Paths whose contents
% are actually identical are marked as synchronized and deleted from the
% list.

% \item The remaining paths are displayed to the user, who then has an
% opportunity to change the default actions and choose actions for
% conflicting paths.

% \item When this process is finished, the selected changes are actually
% propagated between the replicas.

% \item Finally, Unison updates its internal state, marking as
% synchronized all the files for which changes were successfully
% propagated.
% \end{enumerate}

Unison is careful to protect both its internal state and the state of
the replicas at every point in this process.  Specifically, the
following guarantees are enforced:
\begin{itemize}
\item At every moment, each path in each replica has either (1) its {\em
  original} contents (i.e., no change at all has been made to this
path), or (2) its {\em correct} final contents (i.e., the value that the
user expected to be propagated from the other replica).
\item At every moment, the information stored on disk about Unison's
private state can be either (1) unchanged, or (2) updated to reflect
those paths that have been successfully synchronized.
\end{itemize}
The upshot is that it is safe to interrupt Unison at any time, either
manually or accidentally.  [Caveat: the above is {\em almost} true there
are occasionally brief periods where it is not (and, because of
shortcoming of the Posix filesystem API, cannot be); in particular, when
it is copying a file onto a directory or vice versa, it must first move
the original contents out of the way.  If Unison gets
interrupted during one of these periods, some manual cleanup may be
required.  In this case, a file called {\tt DANGER.README} will be left
in your home directory, containing information about the operation that
was interrupted. The next time you try to run Unison, it will notice this
file and warn you about it.]

If an interruption happens while it is propagating updates, then there
may be some paths for which an update has been propagated but which
have not been marked as synchronized in Unison's archives.  This is no
problem: the next time Unison runs, it will detect changes to these
paths in both replicas, notice that the contents are now equal, and
mark the paths as successfully updated when it writes back its private
state at the end of this run.

If Unison is interrupted, it may sometimes leave temporary working files
(with suffix \verb|.tmp|) in the replicas.  It is safe to delete these
files.  Also, if the \verb|backups| flag is set, Unison will
leave around old versions of files that it overwrites, with names like
\verb|file.0.unison.bak|.  These can be deleted safely when they are no
longer wanted.

Unison is not bothered by clock skew between the different hosts on
which it is running.  It only performs comparisons between timestamps
obtained from the same host, and the only assumption it makes about
them is that the clock on each system always runs forward.

If Unison finds that its archive files have been deleted (or that the
archive format has changed and they cannot be read, or that they don't
exist because this is the first run of Unison on these particular
roots), it takes a conservative approach: it behaves as though the
replicas had both been completely empty at the point of the last
synchronization.  The effect of this is that, on the first run, files
that exist in only one replica will be propagated to the other, while
files that exist in both replicas but are unequal will be marked as
conflicting.

Touching a file without changing its contents should never affect whether or
not Unison does an update. (When running with the fastcheck preference set
to true---the default on Unix systems---Unison uses file modtimes for a
quick first pass to tell which files have definitely not changed; then, for
each file that might have changed, it computes a fingerprint of the file's
contents and compares it against the last-synchronized contents. Also, the
\verb|-times| option allows you to synchronize file times, but it does not
cause identical files to be changed; Unison will only modify the file
times.)

It is safe to ``brainwash'' Unison by deleting its archive files
{\em on both replicas}.  The next time it runs, it will assume that
all the files it sees in the replicas are new.

It is safe to modify files while Unison is working.  If Unison
discovers that it has propagated an out-of-date change, or that the
file it is updating has changed on the target replica, it will signal
a failure for that file.  Run Unison again to propagate the latest
change.
\finishlater{There are some race conditions. We should probably talk about them.}

Changes to the ignore patterns from the user interface (e.g., using
the `i' key) are immediately reflected in the current profile.


\SUBSECTION{Caveats and Shortcomings}{caveats}

Here are some things to be careful of when using Unison.

\begin{itemize}
\item In the interests of speed, the update detection algorithm may
  (depending on which OS architecture that you run Unison on)
  actually use an approximation to the definition given in
  \sectionref{updates}{What is an Update?}.

  In particular, the Unix
  implementation does not compare the actual contents of files to their
  previous contents, but simply looks at each file's inode number and
  modtime; if neither of these have changed, then it concludes that the
  file has not been changed.

  Under normal circumstances, this approximation is safe, in the sense
  that it may sometimes detect ``false updates'' but will never miss a real
  one.  However, it is possible to fool it, for example by using
  \verb|retouch| to change a file's modtime back to a time in the past.
  \finishlater{One user---Marcus Mottl---claimed that it could also
  happen if we use
  memory mapped I/O, but this is not clear}

\item If you synchronize between a single-user filesystem and a shared
Unix server, you should pay attention to your permission bits: by
default, Unison will synchronize permissions verbatim, which may leave
group-writable files on the server that could be written over by a lot of
people.

You can control this by setting your \verb|umask| on both computers to
something like 022, masking out the ``world write'' and ``group write''
permission bits.

Unison does not synchronize the \verb|setuid| and \verb|setgid| bits, for
security.

\item The graphical user interface is single-threaded.  This
means that if Unison is performing some long-running operation, the
display will not be repainted until it finishes.  We recommend not
trying to do anything with the user interface while Unison is in the
middle of detecting changes or propagating files.

\item Unison does not understand hard links.

\item It is important to be a little careful when renaming directories
containing {\tt ignore}d files.

For example, suppose Unison is synchronizing directory A between the two
machines called the ``local'' and the ``remote'' machine; suppose directory
A contains a subdirectory D; and suppose D on the local machine contains a
file or subdirectory P that matches an ignore directive in the profile used
to synchronize. Thus path A/D/P exists on the local machine but not on the
remote machine.

 If D is renamed to D' on the remote machine, and this change is
 propagated to the local machine, all such files or subdirectories P
 will be deleted.  This is because Unison sees the rename as a delete and a
 separate create: it deletes the old directory (including the ignored files)
 and creates a new one ({\em not} including the ignored files, since they
 are completely invisible to it).
\end{itemize}



\SECTION{Reference Guide}{reference}{ }

This section covers the features of Unison in detail.

\TOPSUBSECTION{Running Unison}{running}

There are several ways to start Unison.
\begin{itemize}
\item Typing ``{\tt unison \NT{profile}}'' on the command line.  Unison
will look for a file \texttt{\NT{profile}.prf} in the \verb|.unison|
directory.  If this file does not specify a pair of roots, Unison will
prompt for them and add them to the information specified by the profile.
\item Typing ``{\tt unison \NT{profile} \NT{root1} \NT{root2}}'' on the command
line.
In this case, Unison will use {\tt \NT{profile}}, which should not contain
any {\tt root} directives.
\item Typing ``{\tt unison \NT{root1} \NT{root2}}'' on the command line.  This
has the same effect as typing ``{\tt unison default \NT{root1} \NT{root2}}.''
\item Typing just ``{\tt unison}'' (or invoking Unison by clicking on
a desktop icon).  In this case, Unison will ask for the profile to use
for synchronization (or create a new one, if necessary).
\end{itemize}

% \finish{Need to check that the text UI actually works this way.  (It
%   doesn't prompt, for sure, but it should.)}

\SUBSECTION{The {\tt .unison} Directory}{unisondir}

Unison stores a variety of information in a private directory on each
host.  If the environment variable {\tt UNISON} is defined, then its
value will be used as the name of this directory.  If {\tt UNISON} is
not defined, then the name of the directory depends on which
operating system you are using.  In Unix, the default is to use
{\tt \$HOME/.unison}.
In Windows, if the environment variable
{\tt USERPROFILE} is defined, then the directory will be
{\tt \$USERPROFILE$\backslash$.unison};
otherwise if {\tt HOME} is defined, it will be
{\tt \$HOME$\backslash$.unison};
otherwise, it will be
{\tt c:$\backslash$.unison}.
On OS X,
{\tt \$HOME/.unison} will be used if it is present, but
{\tt \$HOME/Library/Application Support/Unison} will be created and used by
default.

The archive file for each replica is found in the {\tt .unison}
directory on that replica's host.  Profiles (described below) are
always taken from the {\tt .unison} directory on the client host.

Note that Unison maintains a completely different set of archive files
for each pair of roots.

We do not recommend synchronizing the whole {\tt .unison} directory, as this
will involve frequent propagation of large archive files.  It should be safe
to do it, though, if you really want to.  Synchronizing just the profile
files in the {\tt .unison} directory is definitely OK.


\SUBSECTION{Archive Files}{archives}

The name of the archive file on each replica is calculated from
\begin{itemize}
\item the {\em canonical names} of all the hosts (short names like
  \verb|saul| are converted into full addresses like \verb|saul.cis.upenn.edu|),
\item the paths to the replicas on all the hosts (again, relative
  pathnames, symbolic links, etc.\ are converted into full, absolute paths), and
\item an internal version number that is changed whenever a new Unison
  release changes the format of the information stored in the archive.
\end{itemize}
This method should work well for most users.  However, it is occasionally
useful to change the way archive names are generated.  Unison provides
two ways of doing this.

The function that finds the canonical hostname of the local host (which
is used, for example, in calculating the name of the archive file used to
remember which files have been synchronized) normally uses the
\verb|gethostname| operating system call.  However, if the environment
variable \verb|UNISONLOCALHOSTNAME| is set, its value will be used
instead.  This makes it easier to use Unison in situations where a
machine's name changes frequently (e.g., because it is a laptop and gets
moved around a lot).

A more powerful way of changing archive names is provided by the
\verb|rootalias| preference.  The preference file may contain any number of
lines of the form:
\begin{alltt}
    rootalias = //\NT{hostnameA}//\NT{path-to-replicaA} -> //\NT{hostnameB}/\NT{path-to-replicaB}
\end{alltt}
When calculating the name of the archive files for a given pair of roots,
Unison replaces any root that matches the left-hand side of any rootalias
rule by the corresponding right-hand side.

So, if you need to relocate a root on one of the hosts, you can add a
rule of the form:
\begin{alltt}
    rootalias = //\NT{new-hostname}//\NT{new-path} -> //\NT{old-hostname}/\NT{old-path}
\end{alltt}
Note that root aliases are case-sensitive, even on case-insensitive file
systems.

{\em Warning}: The \verb|rootalias| option is dangerous and should only
be used if you are sure you know what you're doing.  In particular, it
should only be used if you are positive that either (1) both the original
root and the new alias refer to the same set of files, or (2) the files
have been relocated so that the original name is now invalid and will
never be used again.  (If the original root and the alias refer to
different sets of files, Unison's update detector could get confused.)
%
After introducing a new \verb|rootalias|, it is a good idea to run Unison
a few times interactively (with the \verb|batch| flag off, etc.) and
carefully check that things look reasonable---in particular, that update
detection is working as expected.


\SUBSECTION{Preferences}{prefs}

Many details of Unison's behavior are configurable by user-settable
``preferences.''

Some preferences are boolean-valued; these are often called {\em flags}.
Others take numeric or string arguments, indicated in the preferences
list by {\tt n} or {\tt xxx}.  Some string arguments take the backslash as
an escape to include the next character literally; this is mostly useful
to escape a space or the backslash; a trailing backslash is ignored and is
useful to protect a trailing whitespace in the string that would otherwise
be trimmed.  Most of the string preferences can be given several times;
the arguments are accumulated into a list internally.

There are two ways to set the values of preferences: temporarily, by
providing command-line arguments to a particular run of Unison, or
permanently, by adding commands to a {\em profile} in the {\tt .unison}
directory on the client host.  The order of preferences (either on the
command line or in preference files) is not significant.  On the command
line, preferences and other arguments (the profile name and roots) can be
intermixed in any order.

To set the value of a preference {\tt p} from the command line, add an
argument {\tt -p} (for a boolean flag) or {\tt -p n} or {\tt -p xxx} (for
a numeric or string preference) anywhere on the command line.  To set a
boolean flag to \verb|false| on the command line, use {\tt -p=false}.

Here are all the preferences supported by Unison.  This list can be
  obtained by typing {\tt unison -help}.
\begin{quote}
\verbatiminput{prefs.tmp}
\end{quote}
Here, in more detail, is what they do.  Many are discussed in greater detail
in other sections of the manual.
%
\input{prefsdocs.tmp}


\SUBSECTION{Profiles}{profile}

A {\em profile} is a text file that specifies permanent settings for
roots, paths, ignore patterns, and other preferences, so that they do
not need to be typed at the command line every time Unison is run.
Profiles should reside in the \verb|.unison| directory on the client
machine.  If Unison is started with just one argument \ARG{name} on
the command line, it looks for a profile called \texttt{\ARG{name}.prf} in
the \verb|.unison| directory.  If it is started with no arguments, it
scans the \verb|.unison| directory for files whose names end in
\verb|.prf| and offers a menu (provided that the Unison executable is compiled with the graphical user interface).  If a file named \verb|default.prf| is
found, its settings will be offered as the default choices.

To set the value of a preference {\tt p} permanently, add to the
appropriate profile a line of the form
\begin{verbatim}
        p = true
\end{verbatim}
for a boolean flag or
\begin{verbatim}
        p = <value>
\end{verbatim}
for a preference of any other type.

Whitespaces around {\tt p} and {\tt xxx} are ignored.
A profile may also include blank lines and lines beginning
with {\tt \#}; both are ignored.

When Unison starts, it first reads the profile and then the command
line, so command-line options will override settings from the
profile.

Profiles may also include lines of the form \texttt{include
  \ARG{name}}, which will cause the file \ARG{name} (or
\texttt{\ARG{name}.prf}, if \ARG{name} does not exist in the
\verb+.unison+ directory) to be read at the point, and included as if
its contents, instead of the \texttt{include} line, was part of the
profile.  Include lines allows settings common to several profiles to
be stored in one place.  In \ARG{name} the backslash is an escape
character.

A profile may include a preference `\texttt{label = \ARG{desc}}' to
provide a description of the options selected in this profile.  The
string \ARG{desc} is listed along with the profile name in the profile
selection dialog, and displayed in the top-right corner of the main
Unison window in the graphical user interface.

The graphical user-interface also supports one-key shortcuts for commonly
used profiles.  If a profile contains a preference of the form
%
`\texttt{key = \ARG{n}}', where \ARG{n} is a single digit, then
pressing this digit key will cause Unison to immediately switch to
this profile and begin synchronization again from scratch.  In this
case, all actions that have been selected for a set of changes
currently being displayed will be discarded.


\SUBSECTION{Sample Profiles}{profileegs}

\SUBSUBSECTION{A Minimal Profile}{minimalprofile}

Here is a very minimal profile file, such as might be found in {\tt
  .unison/default.prf}:
\begin{verbatim}
    # Roots of the synchronization
    root = /home/bcpierce
    root = ssh://saul//home/bcpierce

    # Paths to synchronize
    path = current
    path = common
    path = .netscape/bookmarks.html
\end{verbatim}

\SUBSUBSECTION{A Basic Profile}{basicprofile}

Here is a more sophisticated profile, illustrating some other useful
features.
\begin{verbatim}
    # Roots of the synchronization
    root = /home/bcpierce
    root = ssh://saul//home/bcpierce

    # Paths to synchronize
    path = current
    path = common
    path = .netscape/bookmarks.html

    # Some regexps specifying names and paths to ignore
    ignore = Name temp.*
    ignore = Name *~
    ignore = Name .*~
    ignore = Path */pilot/backup/Archive_*
    ignore = Name *.o
    ignore = Name *.tmp

    # Window height
    height = 37

    # Keep a backup copy of every file in a central location
    backuplocation = central
    backupdir = /home/bcpierce/backups
    backup = Name *
    backupprefix = $VERSION.
    backupsuffix =

    # Use this command for displaying diffs
    diff = diff -y -W 79 --suppress-common-lines

    # Log actions to the terminal
    log = true
\end{verbatim}

\SUBSUBSECTION{A Power-User Profile}{powerprofile}

When Unison is used with large replicas, it is often convenient to be
able to synchronize just a part of the replicas on a given run (this
saves the time of detecting updates in the other parts).  This can be
accomplished by splitting up the profile into several parts --- a common
part containing most of the preference settings, plus one ``top-level''
file for each set of paths that need to be synchronized.  (The {\tt
  include} mechanism can also be used to allow the same set of preference
settings to be used with different roots.)

The collection
of profiles implementing this scheme might look as follows.
%
The file {\tt default.prf} is empty except for an {\tt include}
directive:
\begin{verbatim}
    # Include the contents of the file common
    include common
\end{verbatim}
Note that the name of the common file is {\tt common}, not {\tt
  common.prf}; this prevents Unison from offering {\tt common} as one of
the list of profiles in the opening dialog (in the graphical UI).

The file {\tt common} contains the real preferences:
\begin{verbatim}
    # Roots of the synchronization
    root = /home/bcpierce
    root = ssh://saul//home/bcpierce

    # (... other preferences ...)

    # If any new preferences are added by Unison (e.g. 'ignore'
    # preferences added via the graphical UI), then store them in the
    # file 'common' rather than in the top-level preference file
    addprefsto = common

    # Names and paths to ignore:
    ignore = Name temp.*
    ignore = Name *~
    ignore = Name .*~
    ignore = Path */pilot/backup/Archive_*
    ignore = Name *.o
    ignore = Name *.tmp
\end{verbatim}
Note that there are no {\tt path} preferences in {\tt common}.  This
means that, when we invoke Unison with the default profile (e.g., by
typing '{\tt unison default}' or just '{\tt unison}' on the command
line), the whole replicas will be synchronized.  (If we {\em never} want
to synchronize the whole replicas, then {\tt default.prf} would instead
include settings for all the paths that are usually synchronized.)

To synchronize just part of the replicas, Unison is invoked with an
alternate preference file---e.g., doing '{\tt unison workingset}', where the
preference file {\tt workingset.prf} contains
\begin{verbatim}
    path = current/papers
    path = Mail/inbox
    path = Mail/drafts
    include common
\end{verbatim}
causes Unison to synchronize just the listed subdirectories.

The {\tt key} preference can be used in combination with the graphical UI
to quickly switch between different sets of paths.  For example, if the
file {\tt mail.prf} contains
\begin{verbatim}
    path = Mail
    batch = true
    key = 2
    include common
\end{verbatim}
then pressing 2 will cause Unison to look for updates in the {\tt Mail}
subdirectory and (because the {\tt batch} flag is set) immediately
propagate any that it finds.


\SUBSECTION{Keeping Backups}{backups}

When Unison overwrites (or deletes) a file or directory while propagating changes from
the other replica, it can keep the old version around as a backup.  There
are several preferences that control precisely where these backups are
stored and how they are named.

To enable backups, you must give one or more \verb|backup| preferences.
Each of these has the form
\begin{verbatim}
    backup = <pathspec>
\end{verbatim}
where \verb|<pathspec>| has the same form as for the \verb|ignore|
preference.  For example,
\begin{verbatim}
    backup = Name *
\end{verbatim}
causes Unison to keep backups of {\em all} files and directories.  The
\verb|backupnot| preference can be used to give a few exceptions: it
specifies which files and directories should {\em not} be backed up, even if
they match the \verb|backup| pathspec.

It is important to note that the \verb|pathspec| is matched against the path
that is being updated by Unison, not its descendants.  For example, if you
set \verb|backup = Name *.txt| and then delete a whole directory named
\verb|foo| containing some text files, these files will not be backed up
because Unison will just check that \verb|foo| does not match \verb|*.txt|.
Similarly, if the directory itself happened to be called \verb|foo.txt|,
then the whole directory and all the files in it will be backed up,
regardless of their names.

Backup files can be stored either {\em centrally} or {\em locally}.  This
behavior is controlled by the preference \verb|backuplocation|, whose value
must be either \verb|central| or \verb|local|.  (The default is
\verb|central|.)

When backups are stored locally, they are kept in the same
directory as the original.

When backups are stored centrally, the directory used to hold them is
controlled by the preference \verb|backupdir| and the
environment variable \verb|UNISONBACKUPDIR|.  (The environment variable is
checked first.)  If neither of these are set, then the directory
\verb|.unison/backup| in the user's home directory is used.

The preference \verb|maxbackups| controls how many previous versions of
each file are kept (including the current version).

By default, backup files are named \verb|.bak.VERSION.FILENAME|,
where \verb|FILENAME| is the original filename and \verb|VERSION| is the
backup number (1 for the most recent, 2 for the next most recent,
etc.).  This can be changed by setting the preferences \verb|backupprefix|
and/or \verb|backupsuffix|.  If desired, \verb|backupprefix| may include a
directory prefix; this can be used with \verb|backuplocation = local| to put all
backup files for each directory into a single subdirectory.  For example, setting
\begin{verbatim}
    backuplocation = local
    backupprefix = .unison/$VERSION.
    backupsuffix =
\end{verbatim}
will put all backups in a local subdirectory named \verb|.unison|.  Also,
note that the string \verb|$VERSION| in either \verb|backupprefix| or
\verb|backupsuffix| (it must appear in one or the other) is replaced by
the version number.  This can be used, for example, to ensure that backup
files retain the same extension as the originals.

For backward compatibility, the \verb|backups| preference is also supported.
%
It simply means \verb|backup = Name *| and \verb|backuplocation = local|.


\SUBSECTION{Merging Conflicting Versions}{merge}

Unison can invoke external programs to merge conflicting versions of a file.
The preference \verb|merge| controls this process.

The \verb|merge| preference may be given once or several times in a
preference file (it can also be given on the command line, of course, but
this tends to be awkward because of the spaces and special characters
involved).  Each instance of the preference looks like this:
\begin{verbatim}
    merge = <PATHSPEC> -> <MERGECMD>
\end{verbatim}
The \verb|<PATHSPEC>| here has exactly the same format as for the
\verb|ignore| preference (see \sectionref{pathspec}{Path Specification}).  For example,
using ``\verb|Name *.txt|'' as the \verb|<PATHSPEC>| tells Unison that this
command should be used whenever a file with extension \verb|.txt| needs to
be merged.

Many external merging programs require as inputs not just the two files that
need to be merged, but also a file containing the {\em last synchronized
  version}.  You can ask Unison to keep a copy of the last synchronized
version for some files using the \verb|backupcurrent| preference. This
preference is used in exactly the same way as \verb|backup| and its meaning
is similar, except that it causes backups to be kept of the {\em current}
contents of each file after it has been synchronized by Unison, rather than
the {\em previous} contents that Unison overwrote.  These backups are kept
on {\em both} replicas in the same place as ordinary backup files---i.e.
according to the \verb|backuplocation| and \verb|backupdir| preferences.
They are named like the original files if \verb|backupslocation| is set to
'central' and otherwise, Unison uses the \verb|backupprefix| and
\verb|backupsuffix| preferences and assumes a version number 000 for these
backups.

The \verb|<MERGECMD>| part of the preference specifies what external command
should be invoked to merge files at paths matching the \verb|<PATHSPEC>|.
Within this string, several special substrings are recognized; these will be
substituted with appropriate values before invoking a sub-shell to execute
the command.
\begin{itemize}
\item \relax\verb|CURRENT1| is replaced by the name of (a temporary copy of)
  the local variant of the file.
\item \relax\verb|CURRENT2| is replaced by the name of a temporary
  file, into which the contents of the remote variant of the file have
  been transferred by Unison prior to performing the merge.
\item \relax\verb|CURRENTARCH| is replaced by the name of the backed up copy
  of the original version of the file (i.e., the file saved by Unison
  if the current filename matches the path specifications for the
  \verb|backupcurrent| preference, as explained above), if one exists.
  If no archive exists and \relax\verb|CURRENTARCH| appears in the
  merge command, then an error is signalled.
\item \relax\verb|CURRENTARCHOPT| is replaced by the name of the backed up copy
  of the original version of the file (i.e., its state at the end of
  the last successful run of Unison), if one exists, or the empty
  string if no archive exists.
\item \relax\verb|NEW| is replaced by the name of a temporary file
  that Unison expects to be written by the merge program when it
  finishes, giving the desired new contents of the file.
\item \relax\verb|PATH| is replaced by the path (relative to the roots of
  the replicas) of the file being merged.
\item \relax\verb|NEW1| and \relax\verb|NEW2| are replaced by the names of temporary files
  that Unison expects to be written by the merge program when it
  is only able to partially merge the originals; in this case, \verb|NEW1|
  will be written back to the local replica and \verb|NEW2| to the remote
  replica; \verb|NEWARCH|, if present, will be used as the ``last common
  state'' of the replicas.  (These three options are provided for
  later compatibility with the Harmony data synchronizer.)
\item \relax\verb|BATCHMODE| is replaced according to the batch mode of
  Unison; if it is in \texttt{batch} mode, then a non empty string
  (``\verb|batch|'') is substituted, otherwise the empty string is substituted.
\end{itemize}
To accommodate the wide variety of programs that users might want to use for
merging, Unison checks for several possible situations when the merge
program exits:
\begin{itemize}
\item If the merge program exits with a non-zero status, then merge is
  considered to have failed and the replicas are not changed.
\item If the file \verb|NEW| has been created, it is written back to both
  replicas (and stored in the backup directory).  Similarly, if just the
  file \verb|NEW1| has been created, it is written back to both
  replicas.
\item If neither \verb|NEW| nor \verb|NEW1| have been created, then Unison
  examines the temporary files \verb|CURRENT1|  and \verb|CURRENT2| that
  were given as inputs to the merge program.  If either has been changed (or
  both have been changed in identical ways), then its new contents are written
  back to both replicas.  If either \verb|CURRENT1| or \verb|CURRENT2| has
  been {\em deleted}, then the contents of the other are written back to
  both replicas.
\item If the files \verb|NEW1|, \verb|NEW2|, and \verb|NEWARCH| have all
  been created, they are written back to the local replica, remote replica,
  and backup directory, respectively. If the files \verb|NEW1|, \verb|NEW2| have
  been created, but \verb|NEWARCH| has not, then these files are written back to the
  local replica and remote replica, respectively.  Also, if \verb|NEW1| and
  \verb|NEW2| have identical contents, then the same contents are stored as
  a backup (if the \verb|backupcurrent| preference is set for this path) to
  reflect the fact that the path is currently in sync.
  \item If \verb|NEW1| and \verb|NEW2| (resp. \verb|CURRENT1| and
  \verb|CURRENT2|) are created (resp. overwritten) with different contents
  but the merge command did not fail (i.e., it exited with status code 0),
  then we copy \verb|NEW1| (resp. \verb|CURRENT1|) to the other replica and
  to the archive.

  This behavior is a design choice made to handle the case where a merge
  command only synchronizes some specific contents between two files,
  skipping some irrelevant information (order between entries, for
  instance).  We assume that, if the merge command exits normally, then the
  two resulting files are ``as good as equal.'' (The reason we copy one on
  top of the other is to avoid Unison detecting that the files are unequal
  the next time it is run and trying again to merge them when, in fact, the
  merge program has already made them as similar as it is able to.)
\end{itemize}

You can disable a merge by setting a \verb|<MERGECMD>| that does nothing.  For
example you can override the merging of text files specified in a profile by
typing on the command line:
\begin{verbatim}
    unison profile -merge 'Name *.txt -> echo SKIP'
\end{verbatim}

If the \verb|confirmmerge| preference is set and Unison is not run in
batch mode, then Unison will always ask for confirmation before
actually committing the results of the merge to the replicas.

You can detect batch mode by testing \verb|BATCHMODE|; for
example to avoid a merge completely do nothing:
\begin{verbatim}
    merge = Name *.txt -> [ -z "BATCHMODE" ] && mergecmd CURRENT1 CURRENT2
\end{verbatim}

A large number of external merging programs are available.
For example, on Unix systems setting the \verb|merge| preference to
\begin{verbatim}
    merge = Name *.txt -> diff3 -m CURRENT1 CURRENTARCH CURRENT2
                            > NEW || echo "differences detected"
\end{verbatim}
\noindent
will tell Unison to use the external \verb|diff3| program for merging.
%
Alternatively, users of \verb|emacs| may find the following settings convenient:
\begin{verbatim}
    merge = Name *.txt -> emacs -q --eval '(ediff-merge-files-with-ancestor
                             "CURRENT1" "CURRENT2" "CURRENTARCH" nil "NEW")'
\end{verbatim}
\noindent
(These commands are displayed here on two lines to avoid running off the
edge of the page.  In your preference file, each command should be written on a
single line.)

Users running emacs under windows may find something like this useful:
\begin{verbatim}
   merge = Name * -> C:\Progra~1\Emacs\emacs\bin\emacs.exe -q --eval
                            "(ediff-files """CURRENT1""" """CURRENT2""")"
\end{verbatim}

Users running Mac OS X (you may need the Developer Tools installed to get
the {\tt opendiff} utility) may prefer
\begin{verbatim}
    merge = Name *.txt -> opendiff CURRENT1 CURRENT2 -ancestor CURRENTARCH -merge NEW
\end{verbatim}
Here is a slightly more involved hack.  The {\tt opendiff} program can
operate either with or without an archive file.  A merge command of this
form
\begin{verbatim}
    merge = Name *.txt ->
              if [ CURRENTARCHOPTx = x ];
              then opendiff CURRENT1 CURRENT2 -merge NEW;
              else opendiff CURRENT1 CURRENT2 -ancestor CURRENTARCHOPT -merge NEW;
              fi
\end{verbatim}
(still all on one line in the preference file!) will test whether an archive
file exists and use the appropriate variant of the arguments to {\tt
  opendiff}.

Linux users may enjoy this variant:
\begin{verbatim}
    merge = Name * -> kdiff3 -o NEW CURRENTARCHOPT CURRENT1 CURRENT2
\end{verbatim}

Ordinarily, external merge programs are only invoked when Unison is {\em
  not} running in batch mode.  To specify an external merge program that
should be used no matter the setting of the {\tt batch} flag, use the {\tt
  mergebatch} preference instead of {\tt merge}.

\begin{quote}
\it
Please post suggestions for other useful values of the
\verb|merge| preference to the {\tt unison-users} mailing list---we'd like
to give several examples here.
\end{quote}

\finishlater{
\SUBSECTION{Communicating with a Remote Server}{server}

If you can mount both filesystems on the same host, then you can
run with no server (note, though, that this won't be fast enough over
a phone line)..........
}

\SUBSECTION{The User Interface}{ui}

Both the textual and the graphical user interfaces are intended to be
mostly self-explanatory.  Here are just a few tricks:
\begin{itemize}
\item By default, when running on Unix the textual user interface will
try to put the terminal into the ``raw mode'' so that it reads the input a
character at a time rather than a line at a time.  (This means you can
type just the single keystroke ``\verb|>|'' to tell Unison to
propagate a file from left to right, rather than ``\verb|>| Enter.'')

There are some situations, though, where this will not work --- for
example, when Unison is running in a shell window inside Emacs.
Setting the \verb|dumbtty| preference will force Unison to leave the
terminal alone and process input a line at a time.
\end{itemize}

\SUBSECTION{Exit Code}{exit}

When running in the textual mode, Unison returns an exit status, which
describes whether, and at which level, the synchronization was successful.
The exit status could be useful when Unison is invoked from a script.
Currently, there are four possible values for the exit status:
\begin{itemize}
\item [0]: successful synchronization; everything is up-to-date now.
\item [1]: some files were skipped, but all file transfers were successful.
\item [2]: non-fatal failures occurred during file transfer.
\item [3]: a fatal error occurred, or the execution was interrupted.
\end{itemize}
The graphical interface does not return any useful information through the
exit status.

\SUBSECTION{Path Specification}{pathspec}
Several Unison preferences (e.g., \verb|ignore|/\verb|ignorenot|,
\verb|follow|, \verb|sortfirst|/\verb|sortlast|, \verb|backup|,
\verb|merge|, etc.)
specify individual paths or sets of paths.  These preferences share a
common syntax based on regular-expressions.  Each preference
is associated with a list of path patterns; the paths specified are those
that match any one of the path pattern.

\begin{itemize}
\item Pattern preferences can be given on the command line,
  or, more often, stored in profiles, using the same syntax as other preferences.
  For example, a profile line of the form
\begin{alltt}
             ignore = \ARG{pattern}
\end{alltt}
adds \ARG{pattern} to the list of patterns to be ignored.

\item Each \ARG{pattern} can have one of three forms.  The most
general form is a Posix extended regular expression introduced by the
keyword \verb|Regex|.  (The collating sequences and character classes of
full Posix regexps are not currently supported).
\begin{alltt}
                 Regex \ARG{regexp}
\end{alltt}
For convenience, three other styles of pattern are also recognized:
\begin{alltt}
                 Name \ARG{name}
\end{alltt}
matches any path in which the last component matches \ARG{name},
\begin{alltt}
                 Path \ARG{path}
\end{alltt}
matches exactly the path \ARG{path}, and
\begin{alltt}
                 BelowPath \ARG{path}
\end{alltt}
matches the path \ARG{path} and any path below.
%
The \ARG{name} and \ARG{path} arguments of the latter forms of
patterns are {\em not} regular expressions.  Instead,
standard ``globbing'' conventions can be used in \ARG{name} and
\ARG{path}:
\begin{itemize}
\item a \verb|*| matches any sequence of characters not including \verb|/|
(and not beginning with \verb|.|, when used at the beginning of a
\ARG{name})
\item a \verb|?| matches any single character except \verb|/| (and leading
  \verb|.|)
\item \verb|[xyz]| matches any character from the set $\{{\tt x},
  {\tt y}, {\tt z} \}$
\item \verb|{a,bb,ccc}| matches any one of \verb|a|, \verb|bb|, or
  \verb|ccc|.  (Be careful not to put extra spaces after the commas:
  these will be interpreted literally as part of the strings to be matched!)
\end{itemize}
\item
The path separator in path patterns is always the
forward-slash character ``/'' --- even when the client or server is
running under Windows, where the normal separator character is a
backslash.  This makes it possible to use the same set of path
patterns for both Unix and Windows file systems.
\end{itemize}
Some examples of path patterns appear in \sectionref{ignore}{Ignoring
  Paths}.

\SUBSECTION{Ignoring Paths}{ignore}

Most users of Unison will find that their replicas contain lots of
files that they don't ever want to synchronize --- temporary files,
very large files, old stuff, architecture-specific binaries, etc.
They can instruct Unison to ignore these paths using patterns
introduced in \sectionref{pathspec}{Path Specification}.

For example, the following pattern will make Unison ignore any
path containing the name \verb|CVS| or a name ending in \verb|.cmo|:
\begin{verbatim}
             ignore = Name {CVS,*.cmo}
\end{verbatim}
The next pattern makes Unison ignore the path \verb|a/b|:
\begin{verbatim}
             ignore = Path a/b
\end{verbatim}
Path patterns do {\em not} skip filenames beginning with \verb|.| (as Name
patterns do).  For example,
\begin{verbatim}
             ignore = Path */tmp
\end{verbatim}
will include \verb|.foo/tmp| in the set of ignore directories, as it is a
path, not a name, that is ignored.

The following pattern makes Unison ignore any path beginning with \verb|a/b|
and ending with a name ending by \verb|.ml|.
\begin{verbatim}
             ignore = Regex a/b/.*\.ml
\end{verbatim}
Note that regular expression patterns are ``anchored'': they must
match the whole path, not just a substring of the path.

Here are a few extra points regarding the \texttt{ignore} preference.
\begin{itemize}
\item If a directory is ignored, all its descendants will be too.

\item The user interface provides some convenient commands for adding
  new patterns to be ignored.  To ignore a particular file, select it
  and press ``{\tt i}''.  To ignore all files with the same extension,
  select it and press ``{\tt E}'' (with the shift key).  To ignore all
  files with the same name, no matter what directory they appear in,
  select it and press ``{\tt N}''.
%
These new patterns become permanent: they
are immediately added to the current profile on disk.

\item If you use the \verb|include| directive to include a common
collection of preferences in several top-level preference files, you will
probably also want to set the \verb|addprefsto| preference to the name of
this file.  This will cause any new ignore patterns that you add from
inside Unison to be appended to this file, instead of whichever top-level
preference file you started Unison with.

\item Ignore patterns can also be specified on the command line, if
you like (this is probably not very useful), using an option like
\verb|-ignore 'Name temp.txt'|.

\item Be careful about renaming directories containing ignored files.
Because Unison understands the rename as a delete plus a create, any ignored
files in the directory will be lost (since they are invisible to Unison and
therefore they do not get recreated in the new version of the directory).

\item There is also an \verb|ignorenot| preference, which specifies a set of
  patterns for paths that should {\em not} be ignored, even if they match an
  \verb|ignore| pattern.  However, the interaction of these two sets of
  patterns can be a little tricky.  Here is exactly how it works:
  \begin{itemize}
  \item Unison starts detecting updates from the root of the
  replicas---i.e., from the empty path.  If the empty path matches an
  \verb|ignore| pattern and does not match an \verb|ignorenot| pattern, then
  the whole replica will be ignored.  (For this reason, it is not a good
  idea to include \verb|Name *| as an \verb|ignore| pattern.  If you want to
  ignore everything except a certain set of files, use \verb|Name ?*|.)
  \item If the root is a directory, Unison continues looking for updates in
  all the immediate children of the root.  Again, if the name of some child matches an
  \verb|ignore| pattern and does not match an \verb|ignorenot| pattern, then
  this whole path {\em including everything below it} will be ignored.
  \item If any of the non-ignored children are directories, then the process
  continues recursively.
  \end{itemize}
\end{itemize}

\SUBSECTION{Symbolic Links}{symlinks}

Ordinarily, Unison treats symbolic links in Unix replicas as
``opaque'': it considers the contents of the link to be just the
string specifying where the link points, and it will propagate changes in
this string to the other replica.

It is sometimes useful to treat a symbolic link ``transparently,''
acting as though whatever it points to were physically {\em in} the
replica at the point where the symbolic link appears.  To tell Unison
to treat a link in this manner, add a line of the form
\begin{alltt}
             follow = \ARG{pathspec}
\end{alltt}
to the profile, where \ARG{pathspec} is a path pattern as described in
\sectionref{pathspec}{Path Specification}.

Windows file systems do not support symbolic links; Unison will refuse
to propagate an opaque symbolic link from Unix to Windows and flag the
path as erroneous.  When a Unix replica is to be synchronized with a
Windows system, all symbolic links should match either an
\verb|ignore| pattern or a \verb|follow| pattern.


\SUBSECTION{Permissions}{perms}

Synchronizing the permission bits of files is slightly tricky when two
different filesystems are involved (e.g., when synchronizing a Windows
client and a Unix server).  In detail, here's how it works:
\begin{itemize}
\item When the permission bits of an existing file or directory are
changed, the values of those bits that make sense on {\em both}
operating systems will be propagated to the other replica.  The other
bits will not be changed.
\item When a newly created file is propagated to a remote replica, the
permission bits that make sense in both operating systems are also
propagated.  The values of the other bits are set to default values
(they are taken from the current umask, if the receiving host is a
Unix system).
\item For security reasons, the Unix \verb|setuid| and \verb|setgid|
bits are not propagated.
\item The Unix owner and group ids are not propagated.  (What would
this mean, in general?)  All files are created with the owner and
group of the server process.
\end{itemize}


\SUBSECTION{Cross-Platform Synchronization}{crossplatform}

If you use Unison to synchronize files between Windows and Unix
systems, there are a few special issues to be aware of.

\textbf{Case conflicts.}  In Unix, filenames are case sensitive:
\texttt{foo} and \texttt{FOO} can refer to different files.  In
Windows, on the other hand, filenames are not case sensitive:
\texttt{foo} and \texttt{FOO} can only refer to the same file.  This
means that a Unix \texttt{foo} and \texttt{FOO} cannot be synchronized
onto a Windows system --- Windows won't allow two different files to
have the ``same'' name.  Unison detects this situation for you, and
reports that it cannot synchronize the files.

You can deal with a case conflict in a couple of ways.  If you need to
have both files on the Windows system, your only choice is to rename
one of the Unix files to avoid the case conflict, and re-synchronize.
If you don't need the files on the Windows system, you can simply
disregard Unison's warning message, and go ahead with the
synchronization; Unison won't touch those files.  If you don't want to
see the warning on each synchronization, you can tell Unison to ignore
the files (see \sectionref{ignore}{Ignoring Paths}).

\textbf{Illegal filenames.}  Unix allows some filenames that are
illegal in Windows.  For example, colons (`:') are not allowed in
Windows filenames, but they are legal in Unix filenames.  This means
that a Unix file \texttt{foo:bar} can't be synchronized to a Windows
system.  As with case conflicts, Unison detects this situation for
you, and you have the same options: you can either rename the Unix
file and re-synchronize, or you can ignore it.


\SUBSECTION{Slow Links}{speed}

Unison is built to run well even over relatively slow links such as
modems and DSL connections.

Unison uses the ``rsync protocol'' designed by Andrew Tridgell and Paul
Mackerras to greatly speed up transfers of large files in which only
small changes have been made.  More information about the rsync protocol
can be found at the rsync web site (\ONEURL{http://samba.anu.edu.au/rsync/}).

If you are using Unison with {\tt ssh}, you may get some speed
improvement by enabling {\tt ssh}'s compression feature.  Do this by
adding the option ``{\tt -sshargs -C}'' to the command line or ``{\tt
  sshargs = -C}'' to your profile.


\SUBSECTION{Making Unison Faster on Large Files}{speeding}

Unison's built-in implementation of the rsync algorithm makes transferring
updates to existing files pretty fast.  However, for whole-file copies of
newly created files, the built-in transfer method is not highly optimized.
Also, if Unison is interrupted in the middle of transferring a large file,
it will attempt to retransfer the whole thing on the next run.

These shortcomings can be addressed with a little extra work by telling
Unison to use an external file copying utility for whole-file transfers.
The recommended one is the standalone {\tt rsync} tool, which is available
by default on most Unix systems and can easily be installed on Windows
systems using Cygwin.

If you have {\tt rsync} installed on both hosts, you can make Unison use it
simply by setting the {\tt copythreshold} flag to something non-negative.
If you set it to 0, Unison will use the external copy utility for {\em all}
whole-file transfers.  (This is probably slower than letting Unison copy
small files by itself, but can be useful for testing.)  If you set it to a
larger value, Unison will use the external utility for all files larger than
this size (which is given in kilobytes, so setting it to 1000 will cause the
external tool to be used for all transfers larger than a megabyte).

If you want to use a different external copy utility, set both the {\tt
  copyprog} and {\tt copyprogrest} preferences---the former is used for
the first transfer of a file, while the latter is used when Unison sees a
partially transferred temp file on the receiving host.  Be careful here:
Your external tool needs to be instructed to copy files in place (otherwise
if the transfer is interrupted Unison will not notice that some of the data
has already been transferred, the next time it tries).  The default values
are:
\begin{verbatim}
   copyprog      =   rsync --inplace --compress
   copyprogrest  =   rsync --partial --inplace --compress
\end{verbatim}
You may also need to set the {\tt copyquoterem} preference.  When it is set
to {\tt true}, this causes Unison to add an extra layer of quotes to
the remote path passed to the external copy program. This is is needed by
rsync, for example, which internally uses an ssh connection, requiring an
extra level of quoting for paths containing spaces. When this flag is set to
{\tt default}, extra quotes are added if the value of {\tt copyprog}
contains the string {\tt rsync}.  The default value is {\tt default},
naturally.

If a {\em directory} transfer is interrupted, the next run of Unison will
automatically skip any files that were completely transferred before the
interruption.  (This behavior is always on: it does not depend on the
setting of the {\tt copythreshold} preference.)  Note, though, that the new
directory will not appear in the destination filesystem until everything has
been transferred---partially transferred directories are kept in a temporary
location (with names like {\tt .unison.DIRNAME....}) until the transfer is
complete.


\SUBSECTION{Fast Update Detection}{fastcheck}

If your replicas are large and at least one of them is on a Windows
system, you may find that Unison's default method for detecting changes
(which involves scanning the full contents of every file on every
sync---the only completely safe way to do it under Windows) is too slow.
Unison provides a preference {\tt fastcheck} that, when set to
\verb|true|, causes it to use file creation times as 'pseudo inode
numbers' when scanning replicas for updates, instead of reading the full
contents of every file.

When \verb|fastcheck| is set to \verb|no|,
Unison will perform slow checking---re-scanning the contents of each file
on each synchronization---on all replicas.  When \verb|fastcheck| is set
to \verb|default| (which, naturally, is the default), Unison will use
fast checks on Unix replicas and slow checks on Windows replicas.

This strategy may cause Unison to miss propagating an update if the
 modification time and length of the file are both unchanged
by the update.
However, Unison will never {\em overwrite} such an update with a change
from the other replica, since it always does a safe check for updates
just before propagating a change.  Thus, it is reasonable to use this
switch most of the time and occasionally run Unison once with {\tt
  fastcheck} set to \verb|no|, if you are worried that Unison may have
overlooked an update.

Fastcheck is (always) automatically disabled for files with extension
\verb|.xls| or \verb|.mpp|, to prevent Unison from being confused by the
habits of certain programs (Excel, in particular) of updating files without
changing their modification times.

\SUBSECTION{Mount Points and Removable Media}{mountpoints}

Using Unison removable media such as USB drives can be dangerous unless you
are careful.  If you synchronize a directory that is stored on removable
media when the media is not present, it will look to Unison as though the
whole directory has been deleted, and it will proceed to delete the
directory from the other replica---probably not what you want!

To prevent accidents, Unison provides a preference called
\verb|mountpoint|.  Including a line like
\begin{verbatim}
             mountpoint = foo
\end{verbatim}
in your preference file will cause Unison to check, after it finishes
detecting updates, that something actually exists at the path
\verb|foo| on both replicas; if it does not, the Unison run will
abort.

\SUBSECTION{Click-starting Unison}{click}

On Windows NT/2k/XP systems, the graphical version of Unison can be
invoked directly by clicking on its icon.  On Windows 95/98 systems,
click-starting also works, {\em as long as you are not using ssh}.
Due to an incompatibility with OCaml and Windows 95/98 that is not
under our control, you must start Unison from a DOS window in Windows
95/98 if you want to use ssh.

When you click on the Unison icon, two windows will be created:
Unison's regular window, plus a console window, which is used only for
giving your password to ssh (if you do not use ssh to connect, you can
ignore this window).  When your password is requested, you'll need to
activate the console window (e.g., by clicking in it) before typing.
If you start Unison from a DOS window, Unison's regular window will
appear and you will type your password in the DOS window you were
using.

To use Unison in this mode, you must first create a profile (see
\sectionref{profile}{Profiles}).  Use your favorite editor for this.


\appendix
\SECTION{Installing Ssh}{ssh}{ssh}

{\em Warning: These instructions may be out of date.  More current
  information can be found the
  \SHOWURL{http://alliance.seas.upenn.edu/~bcpierce/wiki/index.php?n=Main.UnisonFAQOSSpecific}{Unison
    Wiki}.}

Your local host will need just an ssh client; the remote host needs an
ssh server (or daemon), which is available on Unix systems.  Unison is
known to work with ssh version 1.2.27 (Unix) and version 1.2.14
(Windows); other versions may or may not work.

\SUBSECTION{Unix}{ssh-unix}

Most modern Unix installations come with \verb|ssh| pre-installed.

\SUBSECTION{Windows}{ssh-win}
Many Windows implementations of ssh only provide graphical interfaces,
but Unison requires an ssh client that it can invoke with a
command-line interface.  A suitable version of ssh can be installed as
follows.

\begin{enumerate}
\item Download an \verb|ssh| executable.

Warning: there are many implementations and ports of ssh for
Windows, and not all of them will work with Unison.  We have gotten
Unison to work with Cygwin's port of OpenSSH, and we suggest you try
that one first.  Here's how to install it:
\begin{enumerate}
\item First, create a new folder on your desktop to hold temporary
  installation files.  It can have any name you like, but in these
  instructions we'll assume that you call it \verb|Foo|.
\item Direct your web browser to www.cygwin.com, and click on the
  ``Install now!'' link.  This will download a file, \verb|setup.exe|;
  save it in the directory \verb|Foo|.  The file \verb|setup.exe| is a
  small program that will download the actual install files from
  the Internet when you run it.
\item Start \verb|setup.exe| (by double-clicking).  This brings up a
  series of dialogs that you will have to go through.  Select
  ``Install from Internet.''  For ``Local Package Directory'' select
  the directory \verb|Foo|.  For ``Select install root directory'' we
  recommend that you use the default, \verb|C:\cygwin|.  The next
  dialog asks you to select the way that you want to connect to the
  network to download the installation files; we have used ``Use IE5
  Settings'' successfully, but you may need to make a different
  selection depending on your networking setup.  The next dialog gives
  a list of mirrors; select one close to you.

  Next you are asked to select which packages to install.  The default
  settings in this dialog download a lot of packages that are not
  strictly necessary to run Unison with ssh.  If you don't want to
  install a package, click on it until ``skip'' is shown.  For a
  minimum installation, select only the packages ``cygwin'' and
  ``openssh,'' which come to about 1900KB; the full installation is
  much larger.

  \begin{quote} \em Note that you are plan to build unison using the free
    CygWin GNU C compiler, you need to install essential development
    packages such as ``gcc'', ``make'', ``fileutil'', etc; we refer to
    the file ``INSTALL.win32-cygwin-gnuc'' in the source distribution
    for further details.
  \end{quote}

  After the packages are downloaded and installed, the next dialog
  allows you to choose whether to ``Create Desktop Icon'' and ``Add to
  Start Menu.''  You make the call.
\item You can now delete the directory \verb|Foo| and its contents.
\end{enumerate}
Some people have reported problems using Cygwin's ssh with Unison.  If
you have trouble, you might try other ones instead:
\begin{verbatim}
  http://linuxmafia.com/ssh/win32.html
\end{verbatim}

\item You must set the environment variables HOME and PATH\@.
  Ssh will create a directory \verb|.ssh| in the directory given
  by HOME, so that it has a place to keep data like your public and
  private keys.  PATH must be set to include the Cygwin \verb|bin|
  directory, so that Unison can find the ssh executable.
  \begin{itemize}
  \item
    On Windows 95/98, add the lines
\begin{verbatim}
    set PATH=%PATH%;<SSHDIR>
    set HOME=<HOMEDIR>
\end{verbatim}
    to the file \verb|C:\AUTOEXEC.BAT|, where \verb|<HOMEDIR>| is the
    directory where you want ssh to create its \verb|.ssh| directory,
    and \verb|<SSHDIR>| is the directory where the executable
    \verb|ssh.exe| is stored; if you've installed Cygwin in the
    default location, this is \verb|C:\cygwin\bin|.  You will have to
    reboot your computer to take the changes into account.
  \item On Windows NT/2k/XP, open the environment variables dialog box:
    \begin{itemize}
    \item Windows NT: My Computer/Properties/Environment
    \item Windows 2k: My Computer/Properties/Advanced/Environment
      variables
    \end{itemize}
    then select Path and edit its value by appending \verb|;<SSHDIR>|
    to it, where \verb|<SSHDIR>| is the full name of the directory
    that includes the ssh executable; if you've installed Cygwin in
    the default location, this is \verb|C:\cygwin\bin|.
  \end{itemize}
  \item Test ssh from a DOS shell by typing
\begin{verbatim}
      ssh <remote host> -l <login name>
\end{verbatim}
    You should get a prompt for your password on \verb|<remote host>|,
    followed by a working connection.
  \item Note that \verb|ssh-keygen| may not work (fails with
  ``gethostname: no such file or directory'') on some systems.  This is
  OK: you can use ssh with your regular password for the remote
  system.
\item You should now be able to use Unison with an ssh connection. If
  you are logged in with a different user name on the local and remote
  hosts, provide your remote user name when providing the remote root
  (i.e., \verb|//username@host/path...|).
\end{enumerate}

\SECTION{Changes in Version \unisonversion}{news}{news}

\input{changes.tex}

\finishlater{
\SECTION{Other Synchronizers}{other}{other}

Unison is just one of several file synchronizers that are currently
available.

Check out:
  http://www.bell-labs.com/project/stage/
  I notice a bunch of people are also doing "data vaulting", e.g.,
    http://www.pc.ibm.com/us/thinkpad/datavault.html
  midnight commander??

Also:
  D. Duchamp
  A Toolkit Approach to Partially Disconnected Operation
  Proc. USENIX 1997 Ann. Technical Conf.
  USENIX, Anaheim CA, pp. 305-318, January 1997
}

\finishlater{
\SECTION{TODO}{todo}{ }

Things to write about:
\begin{itemize}
\item When started in 'socket server' mode, Unison prints 'server started' on
  stderr when it is ready to accept connections.
  (This may be useful for scripts that want to tell when a socket-mode server
  has finished initialization.)
\item {\tt DANGER.README}.
\end{itemize}
}

\finishlater{
Things to write about later:
\begin{itemize}
\item Document different reporting of file status when no archives
  were found.
\item Document buttons in graphical UI
\end{itemize}
}

\iftextversion
\SECTION{Junk}{ }{ }
\fi

\ifhevea\begin{rawhtml}</div>\end{rawhtml}\fi

\end{document}

      \end{quote}
    \fi
  \else
    \@opentoc{htoc}
    \tableofcontents
  \fi
}
\makeatother

\newcommand{\SNIP}[2]{%
\ifhevea\iftextversion
\begin{rawhtml}<pre>----SNIP----\end{rawhtml}
#1
#2 %
\begin{rawhtml}</pre>\end{rawhtml}%
\fi\fi
}

\newcommand{\sectionref}[2]{%
\ifhevea
  \iftextversion
    the section ``#2''
  \else
    the \url{##1}{#2} section%
  \fi
\else
  Section~\ref{#1} {[#2]}%
\fi
}

\newcommand{\bcpurl}[1]{\url{#1}}

\newcommand{\urlref}[2]{\bcpurl{##1}{#2}}
\newcommand{\ONEURL}[1]{%
  \iftextversion#1\else{\def~{\symbol{"7E}}\oneurl{#1}}\fi}
\newcommand{\URL}[2]{%
  \iftextversion#2 (#1)\else\bcpurl{#1}{#2}\fi}
\newcommand{\SHOWURL}[2]{%
  \ifhevea\URL{#1}{#2}\else#2\footnote{{\def~{\symbol{"7E}}\tt #1}}\fi}

% Usage: \SECTION{Title and menu item name}{tex label}{man section id}
\newcommand{\SECTION}[3]{%
  \ifhevea
    \SNIP{#1}{#3}%
    \iftextversion\else \@print{<hr>}\fi%
    \section*{\label{#2}#1}%
  \else
    \newpage
    \section{\label{#2}#1}%
    \addtocontents{htoc}{{\string\large\string\bf\string\urlref{#2}{#1}}\\}%
  \fi
}

\newcommand{\SUBSECTION}[2]{%
  \ifhevea
    \subsection*{\label{#2}#1}%
  \else
    \subsection{\label{#2}#1}%
    \addtocontents{htoc}{\hspace{10em}\bullet\string\urlref{#2}{#1}\\}
  \fi
}

\newcommand{\SUBSUBSECTION}[2]{%
  \ifhevea
    \subsubsection*{\label{#2}#1}%
  \else
    \subsubsection{\label{#2}#1}%
    \addtocontents{htoc}{\hspace{18em}\string\urlref{#2}{#1}\\}
  \fi
}

\newcommand{\TOPSUBSECTION}[2]{%
  \ifhevea\SNIP{#1}{#2}\fi
  \SUBSECTION{#1}{#2}%
}

% The quote-based macros looks a imperfect, perhaps due to the lack of
% alignment
% \newenvironment{textui}{{\em Textual Interface:}\begin{quote}}{\end{quote}}
% \newenvironment{tkui}{{\em Graphical Interface:}\begin{quote}}{\end{quote}}
\newenvironment{textui}{\medskip{\em Textual Interface:}\begin{itemize}\item[]
  }{\end{itemize}}
\newenvironment{tkui}{\medskip{\em Graphical Interface:}\begin{itemize}\item[]
  }{\end{itemize}}
\newenvironment{changesfromversion}[1]{%
  \noindent Changes since #1:
  \begin{itemize}
}{
  \end{itemize}
}

\newcommand{\incompatible}{%
  \iftextversion
    INCOMPATIBLE CHANGE:
  \else
    {\bf Incompatible change:}
  \fi}

\newcommand{\UNISONUSERS}{\URL{mailto:unison-users@yahoogroups.com}{{\tt
      unison-users@yahoogroups.com}}}
\newcommand{\UNISONHACKERS}{\URL{mailto:unison-hackers@lists.seas.upenn.edu}{{\tt
      unison-hackers@lists.seas.upenn.edu}}}

\ifhevea
 \makeatletter
 \let\oldmeta=\@meta
 \renewcommand{\@meta}{%
 \oldmeta
\ifdraft
 \begin{rawhtml}
 <META name="Author" content="Benjamin C. Pierce">
 <link rel="stylesheet" href="/home/bcpierce/pub/unison/unison.css">
 \end{rawhtml}
\else
 \begin{rawhtml}
 <META name="Author" content="Benjamin C. Pierce">
 <link rel="stylesheet" href="http://www.cis.upenn.edu/~bcpierce/unison/unison.css">
 \end{rawhtml}
\fi
}
 \makeatother
\fi
